\chapter{Conclusion}
\label{ch.conclusion}

Although this study primarily set out to explore the role of \isi{salience} in \isi{exemplar} \isi{priming} it has also produced a number of related results, which are nonetheless interesting.
The claim that younger Scousers' speech is noticeably more local \parencite[cf.][]{watson2007a} could be confirmed, but only for the two salient\is{salience} variables in the sample (\textsc{nurse}-\textsc{square} and lenition), which appear to carry considerable amounts of covert \isi{prestige}.
Local variants of non-salient\is{salience} variables, on the other hand, were actually found to be receding.
Young Liverpudlians seem to be somewhat more willing to express a local \isi{identity} linguistically than older ones, but they rely almost exclusively on highly salient\is{salience} and/or stigmatise\is{stigmatisation}d features for doing so.

Linguistic norms and \isi{attitude}s in the \isi{speech community} have remained relatively stable.
Speakers of all three generations investigated generally like `soft' or light Liverpool accents, but largely reject very strong ones, to a not inconsiderable degree because the latter are perceived as exaggerated and artificial\is{plastic Scouse(r)}.
Despite these similarities the presence of hypercorrect\is{hypercorrection}ion particularly in the middle-aged speakers suggests that this group is most sensitive to the negative \isi{image} of Liverpool and Scouse, probably because economic decline and \isi{stigmatisation} of the city were at its historic height in the 1970's and 80's when these speakers were growing up.

The phonological variables investigated are not equally salient\is{salience} in all three age groups.
For happ\textsc{y}-tensing and velar nasal plus there is essentially no \isi{change}, both variables are largely below the radar for all speakers in the sample.
With respect to the \textsc{nurse}-\textsc{square} merger, however, conscious\is{awareness} and sub-conscious\is{awareness} aware\is{awareness}ness declines from the middle to the young generation, while lenition of /k/ sees a steady and linear increase in \isi{salience} from the oldest to the youngest speakers.
Crucially, however, lenition of /k/ is the most salient\is{salience} feature in \emph{all} age groups, and is universally followed by the \textsc{nurse}-\textsc{square} merger, velar nasal plus, and happ\textsc{y}-tensing.
While speakers of different age groups have thus not the same level of aware\is{awareness}ness of the individual variables, the relative ordering is the same in all three generations.

This ordering is then mirrored in the perception data.
Both accuracy of `correct' token selection and statistical robustness of the \isi{priming} effect correlate with the social \isi{salience} of the test variable.
No effect at all is detectable for happ\textsc{y}-tensing, and only a weak one for velar nasal plus (if participant as a random factor is not taken into account).
The \textsc{nurse}-\textsc{square} merger and /k/ lenition, on the other hand, both generate robust \isi{priming} effects, and for the latter \isi{salience} can even explain differences between sub-groups of stimuli (divided by phonological environment\is{phonological context}) or subjects (middle- vs. working-class background).
The main hypothesis that this study was built on could thus be confirmed: The more socially salient\is{salience} a linguistic variable is, the more pronounced the resulting effect in an \isi{exemplar} \isi{priming} experiment will be; below a certain level of sub-conscious\is{awareness} aware\is{awareness}ness no statistically significant \isi{priming} effects are generated.

Intriguingly, all significant effects in the perception experiment are in the unexpected direction: Subjects who have been led to believe that the speaker is from Liverpool are \emph{less} likely to perceive variants typical of Liverpool English.
While it is seemingly at odds with existing \isi{priming} research in sociolinguistics, this result is actually compatible with previous work in psychology and suggests that the \isi{phonetic distance} between the prime\is{priming} and the actual speech signal is too great for perceivers to include the stimulus in the prime\is{priming}d category.
Priming\is{priming} works nevertheless, but the outcome is a \isi{contrast effect} instead of the \isi{assimilation effect}s that were found in the studies conducted in Detroit and New Zealand.

Another unexpected outcome of the perception test is that \isi{frequency} of the carrier word is not really a factor worth mentioning when it comes to predicting how subjects will perceive the stimulus.
In chapter \ref{ch.sal} I did argue that \isi{frequency} of \emph{remembrance\is{memory structure}} and not \isi{frequency} of \emph{occurrence} should be most important, but it is still surprising that the latter should essentially play no role at all.
It is possible that \isi{frequency} is just not relevant in this particular context.
The production data support this idea, because \isi{frequency} turned out to be a (nearly) non-significant predictor in production as well.
All the same, a different test design that is specifically aimed at investigating \isi{frequency} effects in \isi{priming} experiments might be able to yield further interesting insights.

For the perception test, it would also be desirable to have a less biased sample of participants than the one this study is based on.
The dataset for perception is quite heavily skewed towards participants that are in their twenties and have a middle-class background.
This is not due to a flaw in design, but something of an unfortunate coincidence linked to the difficulties of recruiting participants over the internet.
A more balanced sample of subjects would, however, enable the researcher to conduct a much more thorough analysis of the impact of social characteristics of the perceivers than I have been able to do in this study.
The tentative results and conclusions presented in this book, and, more importantly, the ones that can be found in previous research \parencite[cf.][]{hayetal2006a,haydrager2010} strongly suggest that this is a fruitful area for future research that can help us to better understand how language perception works.

Turning back to the primary issue of this thesis, my analysis shows that \isi{exemplar} \isi{priming} in sociolinguistics not only needs a variable that comes with a high degree of social \isi{salience}.
In addition, two further requirements have to be met, at least when the goal is to create an \isi{assimilation effect}: 
The \isi{phonetic distance} between the prime\is{priming}d variety and the one actually used in the stimuli must be comparatively small, and \isi{categorisation} of the stimuli must be a comparatively difficult task to start with.
So far, criteria defining contexts where `successful' \isi{exemplar} \isi{priming} is to be expected have been lacking.
I hope that the ones I have suggested here can serve as a starting point for developing a more elaborate `theory of \isi{priming}' \parencite[cf.][]{cesario2014} in the realm of sociophonetics.