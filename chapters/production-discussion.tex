\chapter{Discussion (production)}
\label{ch.prod_discussion}

This chapter will provide a summary and interpretation of the (most important) results reported in Chapters \ref{ch.prod_results_vow}, \ref{prod.res.con}, and \ref{prod.res.qual}.
In line with the primary interest of this thesis, the focus will be on what patterns of usage, distributions across social groups, and explicit comments and attitudes tell us about the status of the variables under scrutiny here: That is to say whether they can best be classified as indicators, markers, or stereotypes.

\section{happ\textrm{\textsc{y}}: Indicator (of northernness)}
\label{prod.disc.happy}

\subsection{Overall age differences}
\label{prod.disc.happy.age}

F1-F2 plots of happ\textsc{y} have shown that this \isi{vowel} is not stable across the three generations of speakers investigated in this study, neither in terms of height nor with respect to \isi{frontness} (though change in the latter is only significant in the raw data, but not once the random effects of individuals and carrier words have been eliminated by a mixed linear effects model).
Rather, realisations of this \isi{vowel} become simultaneously lower and more central from the old to the middle-aged, and from the middle-aged to the young speakers in my sample.
Nevertheless, \isi{Pillai} scores show that happ\textsc{y} and \textsc{fleece} are completely merged for \emph{all} speakers.
Given that the vowels could only be compared in the two (formal) reading tasks this might be expected, because, in such contexts, happ\textsc{y} is more likely to be tense due to phonetic factors such as duration.
It turns out, however, that happ\textsc{y} and \textsc{fleece} are actually moving together: \emph{both} vowels are more central in the middle and the young group.
At the same time, though, the distance between mean realisations of \textsc{fleece} and happ\textsc{y} is increasing in the younger participants, which means that the two vowels are actually becoming more distinct due to happ\textsc{y} being more strongly centralised than \textsc{fleece} --- thus, while both vowels are moving, it does appear to be primarily happ\textsc{y} that is changing.
A general caveat is still in order, because all the differences between age groups are in fact very subtle.
Impressionistically at least, almost all happ\textsc{y} realisations are still acoustically tense, even in the youngest speakers.

Nonetheless, there is a measurable and statistically robust trend for younger speakers to have laxer and therefore less \isi{Scouse} realisations of happ\textsc{y}.
Since these speakers were actually expected to have more \emph{local} pronunciation, an explanation is warranted.
\textcite{flynn2010} found young speakers from Nottingham to employ ultra-\isi{lax} variants of happ\textsc{y} in an attempt to further distance themselves from the south of England and to emphasise their identities as (working class) northerners.
I suspect that young Scousers use laxer happ\textsc{y} variants for the same reason.
Qualitative analysis of comments about identity (cf. \ref{aware_res.north}) revealed that older subjects often consider \isi{Liverpool} to be `unique' or `distinct' from the rest of England, both north and south.
In the middle and particularly the young group, however, having a \isi{secondary identity} as a \isi{northerner} seems perfectly acceptable and even normal to the majority of subjects.
Younger Scousers often readily embrace a northern identity as a means of setting themselves apart from the south and, at the same time, associating with other northern cities that they perceive as (more) similar to \isi{Liverpool}.

As mentioned above, change in happ\textsc{y} is subtle, but it \emph{is} a movement \emph{away} from both the traditional local norm and the modern standard pronunciation (both of which are tense), and \emph{towards} the variant that is typical for the majority of speakers in the linguistic north of England (the only exceptions being \isi{Liverpool} and Newcastle).
Centralising happ\textsc{y} can be seen as a way of linguistically expressing solidarity with other northern cities and keeping one's distance from `the south', a region that many Liverpudlians consider to be both geographically and culturally distant.

\subsection{Gender and class}
\label{prod.disc.happy.social}

The more detailed analysis of happ\textsc{y} in Sections \ref{sec.prod.res.vow.happy.f1} and \ref{sec.prod.res.vow.happy.f2} showed that both gender and \isi{social class} also play a role in how this \isi{vowel} is realised.
For instance, it turns out that the age difference discussed above is exclusively driven by female speakers.
The men in my sample actually all have comparatively low happ\textsc{y} variants, regardless of their age.
Women seem to have been adapting to the men in this respect for quite a while and have now done so to the point that there is no significant gender difference in the youngest generation of speakers any more, which could actually indicate that this variable is slightly more \isi{salient} in the older two generations.

With respect to the front-back dimension women only have happ\textsc{y} variants that are statistically different from those of men when they speak freely and when they imitate a particularly strong \isi{Scouse} accent.
In the former case, women's realisations are fronter, in the latter they are more retracted.
It is not really surprising for women to have fronter happ\textsc{y} variants than men in spontaneous speech, because these fronter realisations are actually closer to the (modern) standard, and numerous sociolinguistic studies have shown that women generally tend to use more standard variants than men.
It does seem strange, however, that they would use more retracted vowels than male speakers when performing \isi{Liverpool} English, given that \isi{stereotypical} \isi{Scouse} should have \emph{tense} happ\textsc{y}.
This could be a hint that women are at least sub-\isi{consciously} aware of the fact that men actually have more central variants than they themselves in spontaneous speech.
If we assume that the typical \isi{Scouser} people think of when they are asked to perform the accent is male (which does not seem too far-fetched, given the negative stereotypes associated with \isi{Liverpool}), then one can interpret women's happ\textsc{y} realisation during accent \isi{imitation} as more `realistic' than `\isi{stereotypical}'.
This argument is rather speculative, but it is striking that women's mean and median F2 during \isi{accent performance} are virtually identical to the values that men have in spontaneous speech (cf. Figure \ref{fig.box.f2w.happy.stylegender} on page \pageref{fig.box.f2w.happy.stylegender}).

Providing a coherent and unifying interpretation of \isi{social class} is even more difficult, because it interacts with gender when trying to predict F1, and age when the focus is on F2.
Women actually use higher happ\textsc{y} variants than men in both the working and the middle class, but in the former the difference is more pronounced than in the latter.
In fact, it is mostly working-class women that stick out.
Middle-class women, middle-class men, and working-class men all have comparatively similar mean F1 values, whereas happ\textsc{y} realisations of working-class women are considerably higher and thus more \isi{Scouse} (cf. Figure \ref{fig.box.f1w.happy.genderclass} on page \pageref{fig.box.f1w.happy.genderclass}).
This result is diametrically opposed to \textcite{flynn2010}'s finding, because in his study (young) working-class women were the ones that drove the change towards ultra-\isi{lax} happ\textsc{y} variants, whereas in my data, these speakers seem to be the ones that have the most tense variants.
The impact of \isi{social class} on F2, as mentioned above, depends on the age of the participant: in the old and the middle-aged group, working-class speakers have fronter vowels, but among the youngest speakers the effect is reversed and working-class Scousers actually have more central variants.

Women having more standard-like realisations is in line with what many sociolinguists have found in many different contexts, but it is unclear why working-class women in particular would have more standard realisations than their middle-class counterparts --- unless they were hypercorrecting, which is not particularly likely for a largely non-\isi{salient} (see below) variable.
What is more, being working class favours tenser pronunciations with respect to F1 for all age groups, but as far as F2 is concerned, this is only true for old and middle-aged speakers.
For the youngest speakers, however, the effect is reversed, and working-class speakers now \emph{dis}favour tense happ\textsc{y} realisations.
The evidence regarding gender and class thus presents itself as rather inconclusive and difficult to interpret.

\subsection{Style shifting and awareness}
\label{prod.disc.happy.style}

When it comes to style shifting the three generations of speakers do not show any significant differences: All speakers use higher and fronter variants of happ\textsc{y} when they read out a word list and also when they perform a \isi{stereotypical} \isi{Scouse} accent (for F2 style differences are statistically less robust).
This is not the pattern that is commonly associated with \isi{Labovian} style shifting, but register does have an impact on how happ\textsc{y} is realised, so an explanation is called for.

The lower F1 and higher F2 values in the word list readings could be explained phonetically (slower and clearer articulation, resulting in more peripheral vowels generally), but this is difficult for \isi{accent performance}, where the same trend (of more peripheral realisations) was observed.
People seem to believe, as explicit comments revealed, that speaking \emph{fast} is a typical feature of \isi{Scouse}, so provided they incorporate this aspect into their \isi{stereotype} performance it would rather favour \emph{laxer} realisations of happ\textsc{y} instead of tenser ones.
In fact, \isi{vowel} durations were somewhat shorter during \isi{imitation} only for the youngest speakers in the sample, the rest had happ\textsc{y} pronunciations of similar length in text reading, spontaneous speech, and accent \isi{imitation} (cf. Table \ref{tab.dur.style.happy}).
In none of the three age groups can \isi{vowel duration} thus be part of the explanation why happ\textsc{y} realisations are tenser during performance of a strong \isi{Scouse} accent --- for the youngest speakers durations would even pull in the opposite direction.

Another interpretation of the U-shaped line in the two relevant graphs is that two different and, in a way, conflicting, speech norms are at work here.
When people read through the list, they converge towards the standard pronunciation, which is /i/, nowadays, whereas when they do the hyper-\isi{Scouse} pronunciations, they tend to use more /i/-like vowels because that is what distinguishes \isi{Scouse} from the directly surrounding accents.
In the `reading' and `free' styles, articulation is a bit more relaxed (with respect to \emph{both} norms) and happ\textsc{y} tends to be lower, possibly simply for reasons of economy.
The approach of two conflicting norms that pull in the same direction might seem slightly unsatisfying, but some sort of very vague \isi{sub-conscious awareness} of happ\textsc{y} as a feature of \isi{Liverpool} English must be assumed if the increase in height and \isi{frontness} in the \isi{imitation} register is to be explained.

Interestingly, \textcite[102]{newbrook1999} also found ``anomalous stylistic patterning'' in West \isi{Wirral} and adds that 
\begin{inparaenum}[(a)]
	\item ``there was a major issue in respect to \emph{norms}'' (my emphasis), and that
	\item ``[t]his applie[d] in particular to happ\textsc{y}'' \parencite[102]{newbrook1999}.
\end{inparaenum}
Part of the problem is certainly that ``the dialectological facts are complex and the interpretation of responses is often debatable'', and also that many subjects seemed to be confused ``as to what the RP form might actually be'', which is his explanation for the fact that the majority of his participants endorsed [i] despite the fact that this was still a non-standard variant in 1980 when he collected his data \parencite[101]{newbrook1999}.
On the basis of my data at least, \emph{conscious} awareness is out of the question: Not a single participant mentioned happ\textsc{y}-tensing as a typical feature of \isi{Liverpool} English, or otherwise commented on it.

\subsection{Classification}
\label{prod.disc.happy.classification}

The analysis of happ\textsc{y} realisations has unearthed a number of features which hint at a certain degree of \isi{salience}: there is some very basic \isi{social stratification}, and there is a certain impact of register.
However, both are less robust for F2, the \isi{vowel} dimension that usually does most of the sociolinguistic work \parencite[cf.][502]{labov2006a}.
Furthermore, social factors are clearly much less important as predictors of formant values (irrespective of whether we are talking about F1 or F2) than they are for \textsc{nurse} (cf. \ref{prod.disc.nurse}), which indicates lower relative \isi{salience} and strongly suggests that the centralisation of happ\textsc{y} is a change from below \parencite[cf.][78]{labov1994}.
No prototypical style shifting is found for happ\textsc{y}, but style differences are clearly not random either.
Since the (somewhat confusing) impact of style is the same in all age groups and can be interpreted as showing at least the beginnings of some sort of awareness, it seems therefore justified to conclude that happ\textsc{y} is somewhere in between an indicator and a marker for all speakers investigated --- with the aside that it might actually be on its way to returning firmly to the status of an indicator with the youngest speakers, given that gender no longer plays a role.

\section{\textrm{\textsc{nurse}}: Marker to stereotype and back again}
\label{prod.disc.nurse}

\subsection{Overall age differences}
\label{prod.disc.nurse.age}

Traditional \isi{vowel} plots of mean \textsc{nurse} and \textsc{square} realisations (pooled across different speaking styles) revealed that the former is (still) more central than the latter for all speakers investigated.
Both vowels do however become higher and fronter from the youngest to the oldest subjects (which means that \textsc{nurse}, in particular, is becoming more \isi{Scouse}, but only with respect to F2).
At the same time the distance between the means decreases, which means that for young Liverpudlians \textsc{nurse} and \textsc{square} are considerably less distinct than for middle-aged and old Scousers.
Just as for happ\textsc{y} and \textsc{fleece}, a caveat is in order here, because, once again, the differences between the two vowels are minute in absolute terms (especially as far as the F1 dimension is concerned), even for the oldest speakers where the distance is greatest.
This idea is corroborated by \isi{Pillai} scores that are universally near 0 and show \textsc{nurse} and \textsc{square} to be almost perfectly merged in any age group and for any style.

All the same, differences between the age groups could be found, even if they were rather subtle in nature.
For one thing, realisations in the old group mostly vary with respect to F1, whereas middle-aged speakers show more variation in F2.
This alone can already hint at a slight increase in \isi{salience} from the old to the middle-aged speakers, because a wider range of F2 values (as the sociolinguistically more important dimension) suggests a potentially higher functional load when it comes to the social meaning of the variables.
Generally speaking, there is less variation in the most formal and the most informal (stereotyped) styles, which shows that speakers seem to be more agreed on the target realisations of the two vowels in these registers.
Crucially, the difference between the styles decreases across the generations, particularly from the middle-aged to the young generation.
The youngest speakers in the sample not only show few differences in-between speaking styles, but they also exhibit a very small degree of variation across the board, even in spontaneous speech.
Both points serve as evidence for the fact that the realisations of \textsc{nurse} and \textsc{square} seem to have largely stabilised in speakers aged between 19 and 29, which speaks for a decrease in \isi{salience}, in particular from the middle to the young generation.

Plots of mean \isi{vowel} realisations also unearthed interesting differences between the age groups in how \textsc{nurse} and \textsc{square} change along the style continuum.
Among the oldest speakers that were interviewed both vowels move to the front during \isi{accent performance}, which is the expected behaviour, particularly for \textsc{nurse}\footnote{This also indicates that even for these speakers the target for a \isi{Scouse} \textsc{nurse} is a front \isi{vowel}, not a central one as some people might suspect given the history of the \isi{merger} in \isi{Liverpool}; cf. \ref{sec.var.vow.nurse}.}.
In the remaining three styles, however, \textsc{nurse} is remarkably stable and it is mostly \textsc{square} that moves --- crucially, this movement is \emph{towards} \textsc{nurse} rather than away from it, which means that the two vowels are actually more instead of less merged the more formal the register.
This is thus a mild case of \isi{hypercorrection}, because by centralising \textsc{square} (which makes it \emph{less} standard) instead of \textsc{nurse} (which would become \emph{more} standard), people are actually moving the `wrong' \isi{vowel}.

Speakers of \isi{Scouse} aged between 30 and 55 also behave as expected when they are asked to perform a \isi{stereotypical} \isi{Scouse} accent: both \textsc{nurse} and \textsc{square} are fronter than in spontaneous speech.
For the reading passage, middle-aged Liverpudlians adjust the vowels in the same way as the old generation.
\textsc{nurse} hardly moves at all, while \textsc{square} is centralised and thus approaches \textsc{nurse}.
When these speakers read out a word list, finally, \textsc{square} is even further back, while \textsc{nurse} actually gets \emph{fronted}.
As a result, \textsc{nurse} ends up fronter than \textsc{square} in this particular speech style.
We have thus a situation that is characterised not only by the fact that the two vowels are more instead of less merged in more formal contexts, but also by a reversal of their relative positioning to each other.
Speakers of the middle generation can therefore be said to present a textbook case of \isi{hypercorrection}, because their behaviour results in the opposite of what they are presumably trying to achieve: \textsc{nurse} and \textsc{square} pronunciations are even more non-standard in formal registers than they already are in spontaneous speech.
This suggests both heightened awareness of the social meaning of this \isi{merger} (hence the urge to modify usage according to communicative situation) and also a certain degree of \isi{linguistic insecurity} with respect to this variable.

Among the youngest speakers style seems to be much less important.
\textsc{nurse} and \textsc{square} are about equally stable across different registers.
This is true both in terms of how big the realisational space is (i.e. the range of occurring variants) and where the centres of gravity of the \isi{vowel} clouds are to be found.
Variation between styles is negligible, the position of both vowels largely constant.
Young speakers have completely merged distributions and almost identical mean realisations in all speech styles, which strongly suggests that \isi{salience} of the \textsc{nurse}-\textsc{square} \isi{merger} is very low at best in this group.

It was also shown that the (age) group \isi{Pillai} scores hide a considerable degree of inter-speaker variation, at least as far as the old and the middle-aged participants are concerned.
These two samples of speakers divide rather neatly into two separate sub-groups:
\begin{inparaenum}[(1)]
	\item Completely merged speakers with \isi{Pillai} scores near 0, and
	\item speakers with comparatively high \isi{Pillai} scores, who keep \textsc{nurse} and \textsc{square} distinct.
\end{inparaenum}
The crucial finding here is that, for the oldest speakers, higher \isi{Pillai} scores correlate with higher social status, because it is the middle class speakers who maintain a distinction and the working class participants who are (more) merged --- just as one would expect in the early phases of the social life cycle of a linguistic variable.
In the middle-aged group there are both middle \emph{and} working class speakers among the merged and the distinct subjects, which shows that awareness has spread to at least some working class speakers (who then try to keep the two vowels more distinct) and also that, in the middle class, speakers have started hypercorrecting, possibly because social awareness (and \isi{stigmatisation}) of this variable has increased for them as well.
When one looks at the young speakers class is no longer an issue at all, because everybody has merged distributions: this echoes the non-impact of style and provides further support to the idea that the \isi{merger} has reached completion and simultaneously dropped completely below the radar again (at least in production).

\subsection{Gender and social class}
\label{prod.disc.nurse.social}

Zooming in on \textsc{nurse} realisations in particular revealed that gender and \isi{social class} interact with age of the participant (and with each other) in a number of ways.
For instance, the mixed linear effects model showed that women had \textsc{nurse} vowels which were significantly higher and fronter than those of men.
As far as F1 is concerned, however, this effect decreases from the old to the young speakers, and is no longer significant for the latter, which is additional evidence for the claim made above that the social \isi{salience} of this variable is lower in the youngest speakers.
Women use higher, i.e. more standard, realisations of \textsc{nurse} in all three age groups, which is what one would expect to find for a socially meaningful variable.
Their values are rather stable across the generations as well, which means that the \isi{apparent time} change in F1 is almost exclusively driven by men, who have raised their \textsc{nurse} to converge with the women in the young group of speakers.

When we look at the front-back dimension, there is no significant gender difference, neither in the oldest nor the youngest speakers.
For the middle group, however, the difference between men and women is not only highly significant, but women actually have \textsc{nurse} realisations that are so much fronter (more \isi{Scouse}) than those of men that the regression model still returns gender as a significant effect although it is only so in this one sub-group.
It would appear strange that women should use more \isi{Scouse} variants of a \isi{salient} variable, but if we \isi{remember} that it is precisely the middle age group that was found to \isi{hypercorrect}, then this actually makes sense.
If a group of speakers is aware of a non-standard feature and so eager to avoid it that they develop a tendency to modify it in the `wrong' direction then it should come as no surprise that that tendency is actually more pronounced for women, given that female speakers are generally held to be more sensitive to linguistic differences that carry social meaning.

Social class has an effect on F2 that is somewhat similar to the one that gender has on F1, albeit in a more moderate way.
Middle-class speakers have more central (standard) \textsc{nurse} variants across all three generations, which is in line with most previous research in sociolinguistics.
However, this difference gets progressively smaller from the oldest to the youngest speakers, which can be seen as further evidence that \textsc{nurse} is decreasing in \isi{salience}, although it has to be said that the class difference is still statistically significant even in the youngest group.
Working class speakers have thus always (within the time frame that is the focus of this study) had very front \textsc{nurse} variants, while middle class speakers have been adapting to this model in the last 50 years or so.

Class and gender of participant interact for both F1 and F2 of \textsc{nurse}, but only in terms of degree, not direction, of effects.
That is to say that, for F1 for instance, the \isi{gender effect} is more pronounced in the middle class (which is to say the distance between the means is greater), but it is highly significant both among middle- \emph{and} working-class subjects.
Interestingly, middle-class speakers of both genders seem to have lower, more \isi{Scouse}, vowels than working-class Liverpudlians.
This is unexpected, but it is not the first time this issue has come up.
After all, one might ask more generally why \textsc{nurse} is consistently shifted upwards throughout each generation (which makes it \emph{less} \isi{Scouse}) while simultaneously being fronted (which makes it \emph{more} \isi{Scouse}).
What might be happening is that fronting of \textsc{nurse} is at least a semi-\isi{conscious} process due to the social importance of the F2 dimension of English vowels, whereas the raising is a change from below that is completely subconscious.
If raising of \textsc{nurse} was a change from below it would not be surprising, but actually \emph{expected} to see (working-class) women in the vanguard \parencite[cf.][292--293]{labov2001a}, as is the case in my sample, where \textsc{nurse} realisations become lower and thus more \isi{Scouse} from working-class women to working-class men, followed by middle-class women and finally middle-class men.
For the front-back dimension of \textsc{nurse} the gender difference is actually somewhat clearer in the working class, but again it should be noted that men and women differ significantly in \emph{both} classes.
Women's higher F2 values have been linked to \isi{hypercorrection} above, and it would not be surprising if this was primarily a feature of the (upper) working class, given that their realisations are, on average, fronter to start with, which could mean that working-class females feel a greater need to `correct' their pronunciation.
As a general note of caution, however, I would like to repeat that the results summarised in this paragraph pertain to rather subtle differences of degree and should not be over-interpreted.

\subsection{Style shifting and awareness}
\label{prod.disc.nurse.style}

When it comes to style shifting there are also some differences between F1 and F2.
In the height dimension, there is little to no style shifting that reaches statistical significance.
If anything, it can be found for the oldest speakers in the sample, but the trend is in the unexpected direction: \textsc{nurse} becomes lower (more \isi{Scouse}) instead of higher in the more formal styles.
When style shifting is investigated for the two genders separately, it turns out that this unexpected trend is actually driven by women of \emph{all} age groups, whereas men exhibit next to no register differences.
A similarly clear distinction is found with \isi{social class}: The downward trend towards less \isi{Scouse} variants the more informal the communicative context is more pronounced for middle class subjects, particularly for the middle age group.
This is again in line with previous research: Female and middle-class speakers exhibiting more style shifting is just what is to be expected for a \isi{salient} variable.
It is true that the shift is in the unexpected direction but this issue has already been discussed above: If raising of \textsc{nurse} is a change from below it \emph{should} actually manifest itself first (and in a more pronounced way) in more informal registers.

Age groups also differ with regard to the impact of style on \isi{frontness} of \textsc{nurse}.
The oldest speakers exhibit almost no style shifting, \textsc{nurse} realisations are only significantly fronter when people imitate a strong \isi{Scouse} accent --- in the other three styles pronunciations are identical from a statistical point of view.
In principle, this holds for both social classes, the differences in style shifting (which is to say the changes between styles, not the absolute values!) are only marginal.
Both points support, once more, the idea that \isi{salience} of this feature is rather low for these speakers.

In the middle-aged group, \textsc{nurse} is significantly less front in free speech than in all the other three styles, which means that the \isi{vowel} does not only become more \isi{Scouse} during performance of a strong accent, but also when people read out a text or a word list.
Again, working- and middle-class speakers behave in a rather similar fashion.
If one only looked at this result in isolation it would be tempting to conclude that there is little style shifting and therefore hardly any awareness of the variable.
However, we know from looking at \textsc{nurse} realisations in relation to \textsc{square} that the middle-age group is actually very prone to \isi{hypercorrection}: they do manipulate \emph{both} vowels in a consistent way, which is just not the expected one; \textsc{nurse} is progressively fronted the more formal the register.
It is this process that is responsible for pronunciations that are comparable in the most formal and the most informal styles, not a lack of \isi{salience}.
The fact that both middle \emph{and} working class speakers \isi{hypercorrect} underlines this by showing that awareness of, and \isi{linguistic insecurity} relating to this \isi{merger} seem to be universal in this age group.

The youngest speakers, finally, have steadily increasing (and significantly different) F2 values from reading out a text to free speech and accent \isi{imitation}.
The only part of their graph which does not look like prototypical style shifting is that \textsc{nurse} is also significantly more front (and thus more \isi{Scouse}) when these speakers read out a word list.
It seems thus as if younger Scousers actually style-shift more consistently than older Liverpudlians, which would be in stark contrast with the evidence discussed in Sections \ref{prod.disc.nurse.age} and \ref{prod.disc.nurse.social}, where I argued that \isi{salience} was \emph{de}creasing for the youngest Scousers.
The contradiction is only apparent, however.
For one thing, the shifting pattern just described is not representative of all speakers in this age group.
Middle- and working-class Scousers aged 29 and younger behave differently, and this difference is not just one of degree.
Rather, working-class speakers contribute the (\isi{hypercorrect}) fronting in the word list style, while the middle-class subjects are responsible for the steep rise of F2 during \isi{accent performance}.
The remaining three styles are not significantly different from each other in both social classes, so taken separately none of them are great style shifters.
In fact, young middle-class Scousers have the flattest line in the sample, i.e. they have a smaller amount of style shifting than any other group (plus there is no significant drop of F2 in free speech due to \isi{hypercorrection} in `reading' and `list').

The other aspect worth considering is that it was shown in Section \ref{sec.prod.res.vow.nurse.pil} (and discussed in \ref{prod.disc.nurse.age}) that young Scousers have the most merged distributions and show the fewest style differences when \textsc{nurse} and \textsc{square} are analysed \emph{together}.
It is true that \textsc{nurse} is somewhat more centralised when these speakers read out a text, but so is \textsc{square}, which means that \isi{vowel} distributions are just as merged (and therefore non-standard) as in the other registers.
We can therefore say that the \isi{salience} of this variable is not gone completely:
Young middle-class Liverpudlians still have at least some \isi{sub-conscious awareness} of fronter \textsc{nurse} variants as a typical feature of \isi{Scouse} (which explains the fronting during performance), while young-working class speakers still \isi{hypercorrect} a bit in the most formal styles (which accounts for the fronting in the word list).
All in all, however, style shifting (and therefore \isi{salience}) is a lot less pronounced in this age group than the relevant line plots suggest, and the impact of style is certainly less than in the middle-aged group.

Explicit comments made by my subjects fit in rather well with people's linguistic behaviour as it has been described and interpreted in this section so far.
Generally speaking, the \textsc{nurse}-\textsc{square} \isi{merger} is not very often commented on (much more rarely than \isi{lenition} of /k/, for instance), but even so there are pronounced differences between speakers of different age groups.
Among the oldest speakers there is hardly any \isi{conscious} awareness of the feature (10\% of speakers comment on it, so only 1 in 10).
In the middle group the percentage of people who explicitly mention the \isi{merger} rises considerably to 38.46\%, only to drop to 13.33\% again in the youngest speakers, a level which is comparable to that of the oldest interviewees.
In contrast to \isi{lenition}, no one, irrespective of their age, singled out the \textsc{nurse}-\textsc{square} \isi{merger} as being a particularly disagreeable or `annoying' feature of \isi{Scouse}.

\subsection{Classification}

Realisations of \textsc{nurse} are governed by a number of social factors, and particularly their interactions.
The impact of these predictors is statistically more robust than for happ\textsc{y}, which is evidenced by the fact that even differences which are very subtle in absolute terms are found to be significant.
This is true even though there are many more observations of happ\textsc{y} than of \textsc{nurse} (and \textsc{square}) in my sample.
All of this suggests a generally higher level of \isi{salience} for \textsc{nurse} in comparison with happ\textsc{y}.

Data on style shifting (particularly when \textsc{square} realisations are also considered) and \isi{conscious} awareness clearly show that the \isi{merger} is not equally \isi{salient} in the three age groups of speakers.
For the oldest speakers it is a marker, awareness of which is only just beginning.
In the middle generations, not only style shifting but also \isi{hypercorrection} is widespread.
Together with a steep increase in \isi{conscious} awareness this shows that the feature is now, for many at least, a \isi{stereotype} that speakers actively (though rather unsuccessfully) try to avoid producing.
Apparently, this boost in \isi{salience} seems to have been only temporary.
Data collected from the youngest speakers in my sample have shown that the \textsc{nurse}-\textsc{square} \isi{merger} has been `reduced' (in terms of social \isi{salience}) to a marker again in the current generation of young adults, possibly even one that might be on its way to becoming an indicator.

\section{Velar nasal plus: Indicator with prestige option}
\label{prod.disc.ng}

\subsection{Age, class, and gender}
\label{prod.disc.ng.social}

Based on accounts in the literature, velar nasal plus was assumed to be one of the less \isi{salient} features of \isi{Scouse}, but realisations were nevertheless found to be influenced by at least some extra-linguistic, i.e. social, factors in interesting ways.
The analysis of age, for instance, revealed that there is a significant increase in the use of velar nasal plus from the old to the middle-aged speakers, which is in line with the idea that \isi{Liverpool} English is getting stronger or more local.
From the middle to the young group, however, there is no further increase.
Rather, \isi{PDF} rates actually drop again.
Compared to the oldest speakers in the sample, Scousers aged between 19 and 29 still use velar nasal plus significantly more often, but if they are judged against the generation of their parents, they cannot be said to have more local realisations of this particular \isi{consonant} as their rates are actually significantly lower.

A closer look at statistical interactions showed that this age difference is actually restricted to middle class subjects.
Only for this socioeconomic class is there a statistically robust rise and subsequent decline of \isi{PDF} from the old to the middle-aged to the young speakers.
For the remaining subjects, on the other hand, the age differences collapse, so in the working class /ŋ(g)/ realisations are actually stable across the timespan investigated in this study.
From the old (where working class and middle class are not significantly different) to the middle age group speakers with higher social status seem to have taken up this variable (i.e. they seem to have become somewhat more aware of it) and increased their usage to a value which is then significantly higher than that of their working class counterparts before subsequently lowering it again a bit so that the classes are, again, no longer statistically distinct among younger Scousers.

Not only \isi{social class}, but also the gender of participant has an impact on how velar nasal plus is used.
The mixed linear effects regression revealed that female speakers have a higher \isi{PDF} (10.85\%), and thus more \isi{Scouse} realisations, of /ŋ(g)/ than men (8.22\%).
With female speakers using more local variants than male ones, we have thus another result which does not seem to resonate very well with previous work in sociolinguistics, but I would like to argue further below that this is only apparently so (see \ref{prod.disc.ng.style}).

Investigation of the significant gender X style interaction also showed that women and men differ with respect to the role style has to play.
Females have a comparatively high \isi{PDF} in the formal styles `word list' and `reading' as well as in `\isi{imitation}', and only reduce this value somewhat in spontaneous speech.
Males, on the other hand, have comparatively low (and statistically identical) values for text reading, free speech, and \isi{accent performance} (although there is a slight rise from `free' to `\isi{imitation}' which is close to significance), and only change their pronunciation (in the same direction as women, i.e. towards more \isi{Scouse} variants) when they are asked to read out a word list.
If one takes spontaneous speech (where there is no significant gender difference) as the baseline it can therefore be said that females seem to be more sensitive to this feature, because
\begin{inparaenum}[(a)]
	\item they change velar nasal plus pronunciations earlier (reading passage) on the way to the formal end of the style spectrum (whereas men need to reach the most formal register before there is any linguistic reaction), and
	\item they react more extremely at the other, most informal, end of the continuum (i.e. \isi{accent performance}), where the rise in \isi{PDF} is much less pronounced for male speakers.
\end{inparaenum}
Both points suggest that women are rather more aware of velar nasal plus than men.

\subsection{Style shifting and awareness}
\label{prod.disc.ng.style}

If the data are pooled across gender and \isi{social class}, no difference between age groups can be found with respect to style shifting.
All speakers, irrespective of their age, have relatively high \isi{PDF} values (\isi{Scouse} realisations containing a \isi{plosive}) when they read out a word list.
There is then a decrease towards `normal' reading style, and a further drop towards spontaneous speech, so from the word list to free speech realisations of velar nasal plus actually become linearly more standard.
From free speech /ŋ(g)/ pronunciations then become considerably more \isi{Scouse} again when subjects are asked to perform a particularly strong \isi{Liverpool} accent (\isi{PDF} is on about the same level as for text reading).
With one negligible exception, all these differences are statistically robust.

While the linear rise from spontaneous speech to text reading to the word list is evidence for some sort of at least subconscious awareness, this is not awareness of velar nasal plus as a \emph{local feature of \isi{Liverpool} English}, because in this case the slope would be wrong.
\isi{PDF} should go \emph{down} in more formal registers, because this would translate to more standard realisations.
The pattern we do actually find therefore rather shows that speakers consider velar nasal plus a characteristic of \emph{careful} speech.
This is unexpected, but actually ties in nicely with the fact that, from a purely synchronic point of view, velar nasal plus is a spelling pronunciation.
Due to its presence in the orthography, it would not be too surprising if speakers considered realising the \isi{plosive} the `proper' way to talk, while not doing so would be a sign of informality.
As outlined above, my data provide further evidence for this interpretation because they show that women have a (very slightly but nevertheless significantly) higher \isi{PDF} than men, which incidentally also echoes \textcite{knowles1973}'s finding that females used velar nasal plus more frequently than males in his sample (cf. \ref{sec.var.con.ng}).
These results are only compatible with many other sociolinguistic studies if we assume that people consider velar nasal plus primarily a feature of careful speech, because then it would actually be expected that women are more prone to using it.
Some additional support for this interpretation can be found in \textcite[cf.][101]{newbrook1999}: In West \isi{Wirral} a not insignificant number of speakers endorsed realisations containing a \isi{plosive} in both \isi{word-final} and particularly in \isi{intervocalic} position, probably because of ``sheer ignorance or confusion as to what the RP form might actually be''.

If velar nasal plus is careful speech, why does its use go up when speakers are asked to perform a strong local accent?
This task was designed to elicit markedly local speech and the evidence pertaining to the other variables (particularly /k/ \isi{lenition}, cf. \ref{prod.res.con.k} and \ref{prod.disc.k.style}) suggests it succeeded.
But of course the accent \isi{imitation} task was still a highly artificial context and speakers presumably paid a lot of attention to their speech, albeit not in the traditional \isi{Labovian} sense of the phrase.
All the same, getting the \isi{stereotype} `right' required them to focus on how they were articulating because this \isi{stereotypical} accent was not their natural one (as is evidenced by the many comments about `falseness', cf. \ref{aware_res.eval}).
It is possible that the increased use of velar nasal plus during \isi{accent performance} is nothing but an artefact of a setting that required subjects to focus very intensely on their pronunciation.
I consider it more likely, however, that in addition to the spelling pronunciation aspect Liverpudlians have at least some awareness of velar nasal plus as a local feature as well.
In this case, the style shifting pattern would again be a result of two conflicting evaluations, or norms, that just happen to pull realisations in the same direction (cf. \ref{prod.disc.happy.style}).

This account involves a certain amount of speculation and, just as for happ\textsc{y}, the issue deserves a much more detailed discussion than the present study can deliver.
Suffice it to say, for the moment, that whatever awareness there is must definitely be subconscious: Not a single subject in the extended secondary dataset (all 38 interviews) mentioned velar nasal plus as a typical feature of \isi{Liverpool} English.

\subsection{Classification}
\label{prod.disc.ng.classification}

Velar nasal plus was originally assumed to be a feature with a rather low amount of \isi{salience} attached to it.
However, its realisations are clearly influenced by social characteristics of the users.
The style dimension, too, is particularly intriguing, and forbids the classification of /ŋ(g)/ as an indicator, since there are clear differences between registers.
At the same time, velar nasal plus is definitely less \isi{salient} than the \textsc{nurse}-\textsc{square} \isi{merger}, which shows both in the lower statistical importance of social predictors (both quantitatively and qualitatively) and the lack of \isi{overt commentary}.
In light of consistent, if somewhat difficult to interpret style-shifting, I conclude, then, that velar nasal plus is a marker for all three age groups investigated and that younger speakers do neither provide evidence for changing \isi{salience} of the feature nor do they, in fact, use the local variant more than their parents' generation.

\section{Lenition: From indicator to stereotype}
\label{prod.disc.k}

\subsection{Age}
\label{prod.disc.k.age}

Among the features investigated in this dissertation, \isi{lenition} of /k/ is the one that generated statistically robust differences for the widest range of social predictors and their combinations.
In fact only frequency of the carrier word was eliminated from the mixed effects regression model; all the other main effects, as well as all interactions that had been entered into the model, turned out to be significant factors in predicting \isi{PDF} values of /k/ (which is why only the most important and relevant results will be discussed here).
Perhaps surprisingly, age of speaker was not among the significant main effects in the regression model, while t-tests on the raw data did find significant age differences, at least between the young speakers and each of the other two groups.
Scousers aged between 56 and 85 have mean \isi{PDF} values comparable to speakers who are between 30 and 55 years old, so from a statistical point of view, and in this particular context, the two groups can actually be considered as one.
Younger Liverpudlians exhibit a significantly higher mean \isi{PDF} than both speakers of their parents' or grandparents' generation.
The apparent contradiction between the raw data and the mixed effects regression has been shown to be mostly due to \emph{like} (as a discourse marker and quotative particle), because, among the youngest speakers, this word is both considerably more frequent and also realised with a higher average \isi{PDF} than in the other two groups. 

This special behaviour of \emph{like} in the young group was filtered out by the mixed effects model since it had a random intercept for carrier word.
While this makes sense in a way (we do not necessarily want a single lexical item to dominate the data in such a way), it also seems somewhat unfortunate.
After all, the fact that young Liverpudlians frequently say [laɪç] (or any other words that they are more likely to realise with a \isi{fricative} than older speakers) probably does contribute considerably to many laypersons' impression that \isi{Scouse} is getting stronger since \isi{lenition} is not only one of the best known but also most stigmatised features (cf. \ref{prod.disc.k.aware}).
Interestingly, the same differences (non-significant between old and middle, but significant between middle and young group) also surface when observations pertaining to \emph{like} are removed from the dataset altogether, so \emph{like} is clearly not the only factor, and young Liverpudlians really do seem to be \isi{Scouser} than those of the middle-aged and old group.

Zooming in a bit reveals that this change has not happened in quite the same way in the two genders.
For women, the increase in \isi{PDF} actually already happens from the old to the middle generation.
From the middle-aged to the young speakers there is then only a slight (and non-significant) further increase of \isi{PDF}, so that young female Scousers do not use \isi{lenition} more than their parents' generation already did.
Male speakers, on the other hand, start out with very high values of \isi{lenition} in the old generation, drop to a considerably and significantly lower level in the middle group, and then increase their usage of \isi{Scouse} variants again from the middle-aged to the young speakers.
With respect to /k/ \isi{lenition}, young men in \isi{Liverpool} have thus completed a sort of revival or `back-to-the-roots' process.

\subsection{Gender and class}
\label{prod.disc.k.social}

Note also that the gender difference is not quite the same within the respective age groups.
It should be noted that the differences are subtle, though: /k/ realisations of women and men are statistically distinct in all three generations of speakers.
However, the difference is slightly less robust in the middle group, and, what is more, women have actually higher \isi{PDF} values than men in this generation.
Generally speaking though, females have \emph{lower} \isi{PDF} means than males, so women use less \isi{lenition} than men, which is just what one would expect for a \isi{salient} and stigmatised variable.
The fact that women are more \isi{Scouse} than men in the middle group might therefore suggest that the variable has lost \isi{salience} in this generation, but additional evidence refutes this hypothesis (see \ref{prod.disc.k.aware}).
Women just seem to have been in the vanguard of this change (remember that their \isi{PDF} has risen systematically from the old to the youngest speakers, whereas the changes in male \isi{PDF} are non-linear), despite the fact that \isi{lenition} is a \isi{salient} non-standard feature, which one would usually rather expect women to shun.
This is indeed a strange result that does not lend itself to straightforward interpretation.
It would be interesting to see whether it is something that just shows up in my sample due to the particular individuals that were recruited or whether it could be replicated and really needs an explanation.

On a different note, it is interesting that the gender difference is not significant when subjects perform a strong \isi{Scouse} accent (where \isi{PDF} is very high in both genders), which can be seen as evidence that both women and men (sub-\isi{consciously}?) consider \isi{lenition} as part of the \isi{Scouse} \isi{stereotype}.
This result is very neatly mirrored when the data are divided according to \isi{social class} of the speaker.
The mixed linear effects regression showed that, all other things being equal, middle class subjects have lower \isi{PDF} rates and thus more standard realisations of /k/ than working class Liverpudlians, which is the expected outcome for a \isi{salient} variable.
Just as with gender, the class difference is highly significant in all speaking styles except accent \isi{imitation} (where \isi{PDF} means are again highest), which shows that the strong association of /k/ \isi{lenition} with a \isi{stereotypical} \isi{Scouse} accent is not only shared among Liverpudlians of both genders, but also across different socioeconomic classes.

However, looking at how the class difference develops across the three generations of speakers investigated here is even more fascinating.
For the oldest speakers, there is actually \emph{no} significant difference in the use of \isi{lenition} between working class and middle class Liverpudlians.
In the middle group the difference is already highly significant, and for the youngest speakers this is even more true.
The reason for this is that, if we take the oldest speakers as the baseline, middle class \isi{PDF} values \emph{de}crease linearly in \isi{apparent time}, whereas working class \isi{PDF} actually \emph{in}creases with the same regularity.
As far as /k/ \isi{lenition} is concerned, the claim that \isi{Scouse} is getting \isi{Scouser} is thus only true for working-class speakers; middle-class realisations of /k/ have actually become more standard in the last few decades.
This final point indicates that the \isi{salience} of this variable has increased among middle-class speakers: they are more aware of \isi{lenition} (and its non-standardness) and therefore try to avoid it.
To explain the opposite trend in the working class one could assume that \isi{salience} in this group has simultaneously decreased, but this is not what is happening (cf. \ref{prod.disc.k.aware}).
Rather, /k/ \isi{lenition} must have acquired \isi{covert} \isi{prestige} as a marker of \isi{local identity}.
For Scousers of lower socioeconomic classes this \isi{covert} \isi{prestige} seems to be more important than the social stigma attached to it, whereas for middle-class speakers priorities are reversed.

\subsection{Style shifting}
\label{prod.disc.k.style}

Style shifting pertaining to /k/ \isi{lenition} reveals highly interesting differences between the age groups, even more so than for \textsc{nurse}.
Old and middle-aged speakers not only have \isi{PDF} values that add up to roughly the same grand mean (cf. \ref{prod.disc.k.age}), they also end up with /k/ pronunciations that are virtually identical in (almost) all individual speech styles analysed in this study.
Neither group has any differences between the registers word list, text reading, and spontaneous speech.
In all three styles /k/ is realised in a comparatively standard-like way.
Only when it comes to accent \isi{imitation} is there a steep rise of \isi{PDF} towards more fricative-like local variants.
These two groups of speakers thus have, at best, a two-way style distinction (\isi{stereotypical} accent vs. everything else), so their awareness of the variable, while not nonexistent, seems to be limited.
Young Scousers, however, presents a textbook case of style shifting as we would expect it for a socially meaningful variable: There is a steady, statistically significant, and almost perfectly linear increase of \isi{PDF} from the most formal to the least formal register.
Compared to the other two groups, the youngest speakers in my sample manipulate \isi{lenition} in a much more fine-grained way, which shows that awareness has reached a level in this group that is considerably higher than for older Liverpudlians.

It has already been pointed out above that women show more awareness of \isi{lenition} than men do.
This higher degree of sensitivity also shows in (slightly) different style shifting patterns.
Female speakers exhibit more systematic and more pronounced style differences than those that are observed for male subjects, so they are not only more sensitive to /k/ \isi{lenition} in general (which would just translate to lower absolute \isi{PDF} values, but not necessarily different style shifting patterns).
With respect to \isi{lenited} variants of /k/ women also are more susceptible to the style dimension.
An additional relevant point here is that when the data are split up along the gender dimension, the youngest speakers are still the ones that show the most systematic style shifting pattern, and this is true for both women \emph{and} men, which shows that the increase in awareness along the age dimension is not limited to just one of the two genders, but is really primarily a question of age.
The fact that style shifting patterns are least different in the young group is further evidence for the idea that awareness of \isi{lenition} is more universal in this generation than in the other two.

The interaction of style, age, and \isi{social class} is somewhat less straightforward to interpret.
First of all, old middle and middle-aged working class subjects have a significant drop in \isi{PDF} from list/reading to spontaneous speech, which means that they use more \isi{Scouse} variants in the more formal registers than in free speech.
This looks very much like the \isi{hypercorrection} that was observed for \textsc{nurse} (cf. \ref{prod.disc.ng.style}), but it is difficult to see why it would affect these two sub-groups in particular.
Generally speaking though, there is a class difference in that middle-class speakers mostly distinguish performance of a strong accent from everything else (in the young group, too), while working-class speakers show more pronounced (and, for the youngest speakers, also more systematic and fine-grained) style differences.

Such a result is unexpected because it could be taken to imply that middle class speakers pay less attention than their working-class counterparts to how often they use stigmatised variants in a particular register.
I believe, however, that this is not the case.
It has to be taken into account that the mean \isi{PDF} values of middle class speakers, especially in the middle and the young generation, are relatively low across the board (accent \isi{imitation} excepted), and considerably lower than even the most standard-like values measured for the working class.
I believe that middle class Liverpudlians just try to avoid \isi{lenited} /k/ variants altogether, even in spontaneous speech, and that they just cannot get much more standard than they already are, even if they wanted to.
They \emph{can} produce \isi{lenited} variants, but they only do so when being asked to reproduce the \isi{stereotype}.
In more natural speech their normal realisation is already so close to their lower (i.e. `standard') limit that there is just no room left for any further style shifting away from the local variant.

\subsection{Awareness and attitudes}
\label{prod.disc.k.aware}

\isi{Lenition} of /k/ attracts, by far, the largest number of explicit comments and evaluations in my sample, which is clear evidence for the fact that it is considerably more \isi{salient} than the other three variables investigated in this thesis.
There are, however, pronounced differences between the three generations.
Among the oldest speakers, \isi{lenition} is hardly mentioned at all, less than one in four subjects mention this feature.
In the middle-aged group this rate has already risen considerably to around 70\%, and when it comes to the youngest Scousers in the sample, each and every one of them expresses \isi{conscious} awareness of \isi{lenition} in velar plosives, so the variable has reached full and universal \isi{stereotype} status.
Many speakers (particularly in the youngest generation) are able to provide comparatively detailed accounts of the phenomenon that include fairly accurate descriptions of the place of articulation and the systematic nature of the variable (as opposed to isolated examples of individual lexical items).

It is not quite clear \emph{why} awareness has increased in the way that it has.
This might be due to the fact that, \isi{impressionistically} at least, contemporary comedians frequently --- and primarily --- use this particular feature to imitate or make fun of the \isi{Liverpool} accent.
One could assume that younger speakers are more exposed to these recent performances than people in, say, their 60s, but at this point this idea is mere speculation.
Furthermore, a separate study would first have to investigate whether performances in the media and on stage really \emph{have} changed in the last decades as to which features they focus on.
After all, it is in principle quite possible that /k/ \isi{lenition} was already part of the external \isi{stereotype} of \isi{Scouse} in the 50s and 60s (it is, for instance, mentioned in the introduction to the first \emph{Lern Yerself Scouse} volume).
We could also argue that increasing awareness in younger speakers is based on the changing usage of the variable.
For instance, people's awareness of non-standard variants might go up when they realise that middle-class speakers increasingly start to avoid them.
However, this would lead to the chicken and egg problem commonly encountered in \isi{salience} research (cf. \ref{sec.sal.sal.circle}): if awareness goes up because usage changes, then what triggers change in usage in the first place?
Whatever the reason for the increase in \isi{conscious} awareness is, explicit comments clearly show that \isi{lenition} of /k/ is even more of a \isi{stereotype} for younger Liverpudlians than it is for the older generations.

Why, then, do younger Liverpudlians use this feature more than their parents or grandparents despite the fact that they know about it?
A different attitude towards \isi{lenition} would be an option, but this is not what we find.
The majority of younger and older speakers alike explained that they themselves often found very strong \isi{Scouse} accents harsh, unpleasant, or intimidating.
\isi{Lenition} of velar plosives is strongly associated with these pronounced \isi{Liverpool} accents and considered one of its most distinctive features.
Across all three generations, subjects judge it rather negatively (`annoying', `makes you \isi{unintelligible}'), possibly because at least some of them have personal experience of outsiders using this variable to make fun of Liverpudlians.
Higher proportions of \isi{lenited} variants among younger Scousers can thus \emph{not} be explained by a different (overt) attitude towards the feature.

Another explanation could be that young speakers just cannot help using \isi{lenition}.
The regular style-shifting that was found, however, rather suggests that particularly younger speakers can at least sub-\isi{consciously} control their usage of the local variant quite well.
It is possible that some Scousers are \emph{generally} aware of the variable but not, for some reason, of its presence \emph{in their own speech}.
There is some anecdotal evidence in my data that supports this idea, like the young female working class speaker who reports not using \isi{lenition} at all when in fact she uses \isi{fricative} variants almost categorically.
It has to be noted, though, that this type of \isi{linguistic insecurity}, while not restricted to this one speaker, does not seem to be the rule (especially with respect to the male speakers).
Just as outlined above for the class differences (cf. \ref{prod.disc.k.social}), I would therefore argue that the higher use of \isi{lenition} among the younger speakers is primarily due to the \isi{covert} \isi{prestige} that \isi{lenited} variants appear to have acquired.

\subsection{Classification}
\label{prod.disc.k.classification}

\isi{Lenition} of /k/ shows precisely the kind of \isi{social stratification} that one would expect a sociolinguistically \isi{salient} variable to produce.
In particular, non-standard realisations are significantly less common among female and middle-class speakers.
Inter-group differences in style shifting suggest that for the old and the middle-aged speakers (sub-\isi{conscious}) awareness is lower than in the young group because the former show only limited style awareness, while the latter present a textbook case of \isi{Labovian} style shifting, with non-standard variants getting consistently and linearly more likely the more informal the communicative context.

Data on explicit comments and judgements further showed that \isi{conscious} awareness increases in a linear fashion from the oldest to the youngest Scousers.
My data thus suggest that \isi{lenition} of /k/ has developed from a (beginning) marker in the old group (where only a minority is \isi{consciously} aware of the variable), to a consolidated marker (for the minority) or \isi{stereotype} (for about three out of four speakers) in the middle generation, and then finally to a fully-fledged and universal \isi{stereotype} in the young group of speakers, where not only \emph{every} speaker knows about the feature but where style shifting is also most consistent and regular.
Liverpudlians aged 29 and younger are therefore not only `more \isi{Scouse}' than their parents and grandparents with respect to this variable, but they are also the ones that are most aware of this feature of \isi{Liverpool} English.

\section{Summary}
\label{prod.disc.summary}

People of all age groups closely link \isi{Scouse} to their \isi{local identity}.
Generally speaking, they are both aware and proud of its \isi{distinctness} as an accent, although some subjects also see this as something that can be problematic, because they know about the negative connotations that a \isi{Scouse} accent can carry, particularly for outsiders.
To a degree, some of these external stereotypes (and their evaluation!) seem to have been internalised, as when Liverpudlians base their list of typical features of their own accent on \isi{stereotypical} performances by outsiders.
On the whole, inside evaluations are often ambivalent: `Light' accents are seen as adding an acceptable amount of local flavour and carrying positive connotations such as down-to-earthness, while extremely `strong' \isi{Scouse} accents receive much less favourable judgements (this aspect is also reported in \citealt[33]{delyon1981}, often because they appear as exaggerated, false, and \isi{inauthentic}.
Old and middle-aged Liverpudlians believe these `exaggerated' accents to be more common among younger speakers, but this verdict is not shared by young Scousers who appear to reject `plastic' accents just as much as the older subjects (interestingly, one subject interviewed by \citeauthor{delyon1981} in 1979 claimed already that some Liverpudlians deliberately `exaggerated' the accent, \citealt[cf.][30]{delyon1981}).

Norms and attitudes towards \isi{Scouse} therefore seem to have remained largely stable across the three generations of speakers investigated in this dissertation.
All the same, there is some evidence that middle-aged speakers seem to be particularly sensitive to the \isi{negative image} of \isi{Scouse}, which shows in the \isi{hypercorrection} that these speakers exhibit for the two \isi{salient} variables \textsc{nurse} and /k/.
Arguably, this is because the formative years of these speakers (the 70s and 80s) coincided with the period when the economic situation and the national image of \isi{Liverpool} as a city was at its historic low.
For the youngest speakers, quite the opposite is true.
While the city is still among the most deprived in the country, the participants in the young group have only ever seen things improving a bit every year.
They know about the \isi{negative image} of their city and their accent, but at least to a certain degree they consider these attitudes to be outdated and unjustified.
Pride in their city and the will to express their \isi{local identity} linguistically seem to be strong enough that the \isi{covert} \isi{prestige} of variables such as the \textsc{nurse}-\textsc{square} \isi{merger} and /k/ \isi{lenition} is at least as (and possibly more) important than the social stigma attached to them.

However, young Scousers are not more \isi{Scouse} in every respect.
Rather, they use (highly) \isi{salient} markers and stereotypes (\textsc{nurse} and /k/ \isi{lenition} in my sample) more often and extensively than their parents or grandparents, because that alone is already enough to convey a strong \isi{local identity} (though it has to be said that \isi{intonation} - \isi{impressionistically} at least - also plays a crucial role here and deserves a study of its own).
Non-\isi{salient} features, on the other hand, are either neglected or even sub-\isi{consciously} used for other purposes: Their non-\isi{salience} allows speakers to use them as a means of expression of a regional identity without noticeably deviating from their more local accent.
Thus, they enable speakers ``to appear outward-looking or more cosmopolitan'' without signalling ``disloyalty to local norms'' or, in particular, ``{}`snobishness'{}'' \parencite[13--14]{foulkesdocherty1999a}.
In my sample, this is precisely what seems to be happening to happ\textsc{y}, which is becoming less instead of more \isi{Scouse} (and thus more `northern') in \isi{apparent time}.

While the four variables analysed in this thesis do not carry identical amounts of social \isi{salience} in the three age groups, their relative ordering is the same, irrespective of speaker age:
\begin{inparaenum}[(1)]
	\item happ\textsc{y} is the least \isi{salient} one in the set, parts of the style differences can be explained phonetically although some sub-\isi{conscious} shifting must also be involved.
	\item Velar nasal plus is very similar to happ\textsc{y}, but the style differences are more pronounced, phonetic reasons are less available as explanatory factors, and some subtle gender differences can be detected, all of which indicates slightly higher social \isi{salience} than for happ\textsc{y}.
	\item \textsc{nurse} shows a more detailed and more robust social distribution than both happ\textsc{y} and /ŋ(g)/, more consistent style shifting, and, most importantly, it attracts at least a small amount of explicit commentary, which is clear evidence for a considerably higher degree of sociolinguistic \isi{salience}.
	\item /k/ \isi{lenition}, finally, is the variable which not only generates the most significant differences in usage among social groups and the most systematic style shifting patterns, but it is also the one that most subjects \isi{consciously} know about (in every generation).
	As the feature that is clearly a \isi{stereotype} and even a \isi{shibboleth} for many, or even most, Liverpudlians it is without a doubt the most socially \isi{salient} variable investigated in the context of this thesis.
\end{inparaenum}