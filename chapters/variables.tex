\chapter{Variables}
\label{ch.var}

	\section{General remarks}\label{sec.var.general}

Whatever the precise details of its evolution, in Liverpool developed what Trudgill calls ``an accent rather more `modern' than that of its hinterland'' \citep[70]{trudgill1999} and that he describes as being ``well known to most British people, and very distinct\is{distinctness}ive''.
For instance, \textcite{montgomery2007} found `Scouse' to be the dialect area most often delimited and labelled by lay participants in a map drawing task.
Scouse also turned out to be the most stigmatise\is{stigmatisation}d of the language varieties mentioned by said participants \citep[cf.][194 and 254]{montgomery2007}.
Furthermore, participants provided more linguistic characteristics for Scouse than for any other dialect area, indicating that Scouse (along with Geordie) has a higher cultural \isi{salience} than most other varieties in England.
Subjects commented on a wide array of (stereotypical\is{stereotype}) features, including the lexicon (`calm down'), prosody (`sing song') and phonetics \citep[cf.][180--181]{montgomery2007a}.
\textcite[15]{crowley2012} also emphasises the \isi{salience} of Scouse when he writes that ``(\dots) in Britain and Ireland\is{Irish} at least, Liverpool and Liverpudlians are most widely recognized by their association with a distinct\is{distinctness} form of spoken language``.

Scouse is ``essentially based on [the accents] of the surrounding areas and has many similarities with those of the Central \isi{Lancashire} and Northwest Midlands areas (\dots)'' \citep[70]{trudgill1999}.
Thus, it generally belongs to the northern branch of English English, without being a prototypical specimen.
\citet[18]{wales2006} writes that \isi{Merseyside} is a ```transition' [zone] between Northern and Midland dialect speech'' and \citet[72]{trudgill1999} claims that Scouse is in some respects as southern as it is northern.
Much of its distinct\is{distinctness}iveness is due to phonetic rather than phonemic divergence from the surrounding varieties. \textcite{knowles1973} describes Scouse as being phonologically North(west)ern but phonetically Anglo \isi{Irish} (\citealt[cf. also][80]{knowles1978}; but see Section \ref{sec.hist.19} concerning \isi{Irish} dominance in the dialect mix\is{new-dialect formation}).

The only comprehensive description of Scouse as a whole so far is \cite{knowles1973}, which is based on interview data from two Liverpool electoral wards --- Aigburth to the south and Vauxhall to the north of Liverpool city centre.
At least from the perspective of the time of writing, there are a number of difficulties with Knowles' account.
Parts of his thesis are based exclusively on native speaker introspection (for instance the whole section on what he calls ``setting and voice quality'', \citealt[cf.][102]{knowles1973}).
Also, he seems to embrace some rather strange notions for a linguist, e.g. he claims that ``no-one with any local knowledge would attempt to [make quantitative statements about Liverpool speech in general]'' since ``no sample, however unbiased, would allow one to make inferences about the Chinese and coloured communities of Liverpool 8, or of the University people of Abercromby'' \citep[3]{knowles1973}.
It is not clear whether he thinks this is because he interviewed people from only two electoral wards (which would be fairly obvious and not really worth pointing out) or because he really thinks that for some reason it is not possible to have a representative sample of Liverpool speech in general (which would be an odd thing to say, especially for a sociolinguist).
Occasionally, he even slips into clearly prescriptivist vocabulary, for instance when he describes the voice quality of Scouse as being ``undeniably poor and ugly, as these terms are normally understood'' \citep[116]{knowles1973}.

That said, Knowles is aware of some of these shortcomings, calling his description of the Scouse vowel system ``admittedly speculative'' and ``put forward extremely tentatively'' \citep[111]{knowles1973}.
He also explains that --- originally having intended to apply Labovian methods in his thesis --- he found it problematic to identify and analyse socially significant variables in Scouse, and, consequently, he himself does not consider his study ``a contribution to socio-linguistics as such'' \citep[cf.][1]{knowles1973}.
Notwithstanding these problems, his work is, as mentioned above, the most complete description of Scouse available and any study concerned with the variety of Liverpool must start out from \citeauthor{knowles1973}' PhD thesis.
In the general overview of Scouse characteristics that follows, this project will do the same.
The four variables subjected to closer analysis in this study are discussed in more detail in Sections \ref{sec.var.con.ng}, \ref{sec.var.con.len}, \ref{sec.var.vow.happy}, and \ref{sec.var.vow.nurse} respectively.

	\section{Supragsegmentals}\label{sec.var.supra}

\citet{knowles1973} talks at length about Scouse \isi{intonation} and indeed it is a feature which rather quickly strikes the outsider when first talking to a Liverpudlian.
Several of my own participants (see Section \ref{sec.qual.supra}) also mentioned ``a lilt'' as one of the distinguishing characteristics.
\citeauthor{wales2006} remarks that although ``[supra-segmentals] are such readily distinct\is{distinctness}ive \isi{marker}s of regional origin (\ldots) they have been quite seriously under-researched'' \citeyearpar[201]{wales2006}.
This is certainly true.
However, suprasegmentals are not the focus of this study either, so suffice it to say that ``[t]he \isi{intonation} of Liverpool speech differs notably in some respects from that in England as a whole'' but that ``[e]xactly how much they differ is not easy to assess'' \citep[221]{knowles1973} and sometimes more a matter of relative \isi{frequency} than real difference \citep[cf.][176]{knowles1973}.

According to Knowles, at least working-class \isi{intonation} is ``undoubtably Celtic in origin'', with ``\isi{Irish} influence [being] much more likely than \isi{Welsh}'' \citep[221--222]{knowles1973} and ``the origin of middle class \isi{Merseyside} \isi{intonation} [being] more obscure'' \citep[222--223]{knowles1973}.
Just as for the segments, he claims that Liverpool \isi{intonation} is, in general, ``phonologically North-Western English, but largely phonetically Anglo-\isi{Irish}'' \citep[225]{knowles1973} --- a claim that has to be based on a `phonology of \isi{intonation}', which indeed he sketches in his thesis.
The reader is referred to \citet[174--226]{knowles1973} for details.

Voicing, says \citeauthor{knowles1973}, is ``relatively slow to start up at the beginning of an utterance, and tends to die away just before the end'' \citeyearpar[246]{knowles1973}, meaning that voiced and voiceless sounds are mostly distinguished by the duration of the preceding sound --- which is in fact the most important cue in English \citep[cf., for instance,][]{hoganrozsypal1980}. \citeauthor{knowles1973} claims that ``Scouse differs markedly from the rest of North Midland English'' in this respect and ``is not quite the same as RP'' \citeyearpar[246]{knowles1973}, although he can only be talking about voicing starting rather late, since he --- correctly --- says elsewhere that RP has \isi{devoicing} (in final stops) as well \citeyearpar[cf.][114]{knowles1973}.

	\section{Consonants}\label{sec.var.con}

The repertoire of Scouse consonants is ``phonologically identical to most other varieties of English English'' \citep[351]{watson2007} but the phonetic realisation is often not.
Just like the \isi{Lancashire} dialects it is derived from, Scouse was still rhotic in the 19\textsuperscript{th} century, but it has now lost all traces of this rhoticism \citep[cf.][149]{knowles1997} and is just like RP in this respect.
\emph{Pre}-vocalic /r/ is often realised as a flap in broad Scouse --- especially in intervocalic\is{phonological context} position, but also in onset clusters (cf. \citealt[107 and 329--330]{knowles1973}; \citealt[352]{watson2007}).
Contrary to RP, however, the realisation as [ɾ] is ``a non-\isi{prestige} feature in Liverpool'' and therefore avoided by middle-class speakers \citep[329]{knowles1973}.

/θ/ and /ð/ can be both realised as ``RP-type interdental fricatives [θ ð]'' or as ``Anglo-\isi{Irish} [T, D] which can be post-dental or (apico-)alveolar stops'' \citep[323]{knowles1973}.
\citeauthor{knowles1973} found the realisation as stops being ``virtually restricted (\ldots) to working class Catholics'' and more frequent among men than women \citep[323--324]{knowles1973}.
\cite{watson2007}'s female working-class speaker uses dental stops in all positions and, interestingly, shows no signs of TH-fronting, ``despite the evidence that suggests it is diffusing throughout much of the rest of the country'' \parencite[cf.][352]{watson2007}.

		\subsection{/ŋ(g)/}\label{sec.var.con.ng}

Another characteristic consonantal feature of Scouse is what is often termed `velar nasal plus'.
Most varieties of English pronounce word-final\is{phonological context} <ng> clusters as [ŋ].
The original realisation --- as ``reflected in the spelling which we still use'' \citep[58]{trudgill1999} ---, however, was [ŋg]. In ``Central \isi{Lancashire}, \isi{Merseyside}, Northwest Midlands and West Midlands'' \citep[58]{trudgill1999} this older pronunciation prevails to this day. 
The area in which velar nasal plus is ``a defining characteristic'' \citep[58]{trudgill1999} contains the cities of Birmingham, Manchester, and Liverpool.
In these places, \emph{singer} is not pronounced [sɪŋə] but [sɪŋgə], and \emph{long} is realised as [lɒŋg] instead of [lɒŋ] \citep[cf.][58]{trudgill1999}.

Talking about Scouse in particular, \textcite[293]{knowles1973} describes [g] as ``always optional'' in <ng> clusters, provided it is not obligatory in RP (e.g. in words such as \emph{longer} or \emph{stronger}).
He suggests that [g] is primarily realised word finally or prevocalically, and that [ŋg] ``would be odd'' \parencite[293]{knowles1973} preceding another plosive such as in \emph{stringed}.
The \emph{ing}-forms can also be realised with an audible [g], resulting in [ɪŋg].
According to \textcite[293]{knowles1973}, ``[r]eduplicated /ɪŋg/-forms as in \emph{singing} /sɪŋgɪŋg/'' are possible, but comparatively rare \citep[cf.][293]{knowles1973}.
This is probably mostly due to the fact that, just like in many other places of the English-speaking world, -\emph{ing} is often realised as [ɪn] in Liverpool --- \textcite[cf.][156]{knowles1973} states that this is more frequently so for the present participle than the gerund.

If <ng> occurs word finally ``it can be difficult to decide whether there is a final /g/ or not''.
In these instances, \citeauthor{knowles1973} argues, the length of the preceding nasal, rather than the acoustics of the [g] itself, seems to be an essential cue for perceiving ``/ŋg/ rather than /ŋ/''.
This leads \citeauthor{knowles1973} to the somewhat strange statement that some cases of <ng> ``sound like the Scouse /ŋg/ rather than the standard /ŋ/'', although there is ``no audible /g/'' \citep[293]{knowles1973}.
This does seem odd, since the presence of [g] is the very essence of the Scouse variant.
Note, however, that \citealt{knowles1973} is purely based on auditory analysis --- in the cases described by \citeauthor{knowles1973} there might very well have been some subtle acoustic cues of a `proper' [g] that would have been revealed by methods of phonetic analysis not widely available at the time.

The more or less voluntary realisation of velar nasal plus aside, \cite{knowles1973} also presents another theory of how [ŋg] can come about in final position.
He claims that due to the ``phonation pattern by which voice trails off before the end'' (cf. Section \ref{sec.var.supra}), the (often audible) ``release of the velar closure (\ldots) sounds exactly like a weak oral [g]'', because nasal resonance has stopped \citep[cf.][294]{knowles1973}.
For words such as \emph{anything, something, nothing} (but strangely not in the simple \emph{thing}), \textcite[cf.][156]{knowles1973} also found the realisation [θɪŋk] , combining velar nasal plus with final \isi{devoicing} (again, cf. Section \ref{sec.var.supra}).

Interestingly, \citeauthor{knowles1973} reports that in the (mostly middle-class) district of Aighburth, the majority of the men he interviewed used [ŋ], whereas most women used [ŋg] \citeyearpar[cf.][295]{knowles1973} --- a reversal of the familiar pattern revealed in countless sociolinguistic studies since then, according to which local forms are more common in \emph{male} speech, whilst women tend to use more standard variants.
\citet[352]{watson2007} also reports velar nasal plus --- including reduplicated instances as in \emph{singing} [sɪŋgɪŋg] --- as a characteristic of Liverpool English (his data are taken from the speech of a 21-year-old), so apparently it is not a feature that has disappeared since the 1970s when \citeauthor{knowles1973} published his thesis.

Despite the hints in \textcite{knowles1973} that the use of [ŋg] variants might be socially stratified in Liverpool, at least with respect to gender, velar nasal plus is not counted among the salient\is{salience} features of Scouse.
\textcite[98]{newbrook1999} reports the spread of [ŋg] variants into West \isi{Wirral}, i.e. to the other side of the river Mersey (which is a very salient\is{salience} natural border for many people in the area).
Realisations containing a velar plosive occurred frequently, both in intervocalic\is{phonological context} and in word-final\is{phonological context} contexts.
The majority of speakers did not exhibit any \isi{style shifting} with this variable (although \isi{marker} patterning did occur for some of them), which ``suggests limited \isi{salience}'' of this variable in the wider Liverpool region \parencite[98]{newbrook1999}.

		\subsection{Lenition (of /k/)}\label{sec.var.con.len}

\textcite[251]{knowles1973} explains that in Liverpool English there is an ``apparent confusion of stops, plosives, affricates [and] fricatives (\ldots)'', which he attributes to a general Scouse tendency towards `lax' articulation, resulting in incomplete blocking of the air stream during the closure phase of stops \parencite[cf.][107]{knowles1973}.
The technical term is lenition, from Latin \emph{lenis}, which describes a process of phonological `weakening' along a certain trajectory.
As so often, there is some disagreement about the use of the term \parencite[cf.][196]{watson2002}. For the purposes of this study, I will adhere to \citeauthor{honeybone2007}'s definition as a ``synchronic, variable process whereby underlying plosives are realised as affricates and fricatives in certain specific prosodic and melodic environments''.
He counts this process among ``the clearest phonological characteristics of Modern Liverpool English'' \citeyearpar[129]{honeybone2007}.
All plosives can be subject to lenition in Liverpool English \citep[cf.][236]{honeybone2001}, but most research so far has focused on /t/ and /k/ (see, e.g., \citealt{honeybone2001, sangster2001, watson2002, watson2006}).
According to \textcite[236]{honeybone2001}, the possible realisations (from least lenited to most lenited) are [t, tθ/ts, θ/s, h, ∅] for /t/, and [k, kx, x, h, ∅] for /k/.

In Liverpool, all of the lenited variants of /t/ that are possible actually occur (in various \isi{phonological context}s), but for /k/ only the realisations [kx] and [x] are attested \parencite[cf][242]{honeybone2001}.
It should be added that the fricative realisation of /k/ is not always [x] --- [ç] is also possible.
The two allophones are in complementary distribution for most speakers, and phonologically conditioned: [ç] follows high front monophthongs and raising diphthongs (\emph{week} [wiːç], \emph{like} [laɪç]), whereas velar (or uvular) fricatives occurs in the remaining contexts (\emph{back} [bax], \emph{dock} [dɒχ], \citealp[cf.][353]{watson2007}).
As a result of this process, words such as \emph{matter} and \emph{lock} can sound more like [mæsə] and [lɒkx] or [lɒx], in the last case forming a pair of homophones with the Scots word \emph{loch} \citep[cf.][73]{trudgill1999}.
Note that \cite{knowles1973} talks about an \emph{apparent} confusion, though, hinting at the fact that, while becoming more alike, a phonologically plosive sound does not usually merge completely phonetically with the respective affricate or fricative.
At least as far as the alveolar plosives are concerned the three `cardinal' categories nevertheless remain distinct\is{distinctness} \parencite[cf.][327 and 252--253]{knowles1973}.

Based on his \citeyear{knowles1973} data, \citeauthor{knowles1973} found that the majority of Liverpudlians used ``stops with incomplete closure'' at least every now and then and many apparently even realised lenited stops in rather formal speaking styles.
He therefore concludes that lenition, though originally probably a working-class feature, has also taken hold in middle-class speech.
He nevertheless finds that --- not surprisingly --- lenited variants are more frequent in working-class speech and, with respect to /t/ at least, are also more common among women.
This relates back to Section \ref{sec.var.con.ng} in that it represents another deviation from the common gender pattern \citeyearpar[cf.][325--327]{knowles1973}.

The \isi{frequency} of the individual variants depends mostly on the phonological environment\is{phonological context}, with, for instance, the fricatives being most frequent in ``word-final\is{phonological context} and foot-medial positions'', while other contexts are inhibitive to the use of lenited variants (\citealp[cf.][130]{honeybone2007}; for a discussion of inhibiting environments see \citealt{honeybone2001}).
Especially in intervocalic\is{phonological context} environments lenition is phonetically motivated, which is the reason why it occurs frequently in this context, both in typological terms and in Liverpool English in particular \parencite[cf.][230 and 243]{honeybone2001}.

The history of lenition is more complex than that of other features.
\textcite{hickey1996} claims that lenition was first transferred from \isi{Irish} Gaelic to \isi{Irish} English and then taken to Liverpool by the \isi{Irish} migrants in the 19\textsuperscript{th} century.
The problem with this account, according to \citet{honeybone2007}, is that the patterning of Liverpool lenition is not the same as that of the `initial mutations' in \isi{Irish} Gaelic.
As the name implies, the latter only occur in morpheme-initial segments, whereas lenition in Scouse --- though possible and not infrequent in initial position --- is much more typical word-medially and -finally\is{phonological context}.
What is more, glides and nasals are also affected in Gaelic, but only stops are lenited in Liverpool English \citep[cf.][131]{honeybone2007}.
The \emph{t}-spirantisation attested in southern varieties of \isi{Irish} English that turns /t/ into [θ] is very similar in patterning but still ``distinct\is{distinctness} from the affrico-spirantisation of Liverpool lenition'' \citep[132]{honeybone2007}.
\citeauthor{honeybone2007} concludes that

	\begin{quote}
		(\ldots) the small amounts of plosive lenition that do exist in current forms of Hiberno-English provided some push towards spirantisation, along with the other minor affrications or spirantisations in the input dialects, and that these were developed, following an endogenous pathway of \isi{change}, by those who formed Liverpool English \citeyearpar[131]{honeybone2007}.
	\end{quote}

At least parts of the lenition processes in Liverpool are thus ``endogenously innovated'' \citep[130]{honeybone2007} and the phenomenon was not an `off the shelf' feature readily available in one or several of the varieties that contributed to the formation of Scouse (unlike, for instance, non-rhoticity or the realisation of /θ, ð/ as `Anglo-\isi{Irish} stops').
There was clearly influence from \isi{Irish} English and maybe also some other varieties such as London English which in its present form contains a certain amount of \emph{t}-affrication and might have done so in the 19\textsuperscript{th} century already \parencite[cf.][132]{honeybone2007}.

``The full patterning of Liverpool lenition'', however, constitutes ``a creative act'', performed by ``the young generations of young Liverpudlians who were forming or focusing the koine'' \parencite[132]{honeybone2007}.
It was thus not the result of levelling\is{dialect levelling} towards one of the input varieties but ``a novel, divergent development'' \parencite[132]{honeybone2007}.
As a result, the Scouse type of lenition is not only special in its precise patterning, but also ``unique among varieties of English in its extent'' \parencite[132]{honeybone2007}.
\textcite[130]{honeybone2007} explains that although spirantisation and affrication processes are not unknown in other forms of English, ``no other (\ldots) variety exhibits so much'' \parencite[130]{honeybone2007}.
This is certainly one of the main reasons for the very high \isi{salience} of the feature and its being part of the Scouse \isi{stereotype}.
In the case of /k/, which this book will focus on, this is clearly aided by the fact that [x] is extremely rare among English varieties.

Somewhat surprisingly, /k/ lenition does not figure prominently in what \textcite{honeybonewatson2013} call the `Contemporary Humorous Localised Dialect Literature' (essentially the \emph{Lern Yerself Scouse} series).
A possible explanation is that it is not a straightforward task to represent [x] with the help of the ordinary Latin alphabet.
This cannot be the only reason, however, since lenition in other stops is also not represented in these booklets, despite the fact that there are orthographic representations for doing so.
\textcite{honeybonewatson2013} hypothesise that ``speakers are not very clearly aware\is{awareness} of the existence of the phenomenon'' because it is (a) a comparatively recent, and (b) a sub-phonemic feature which does not entail the collapse of categories \parencite[cf.][329--331]{honeybonewatson2013}.
Their conclusion is that /k/ lenition is ``non-salient\is{salience}'' \parencite[333]{honeybonewatson2013}, but it should be noted that most of the Scouse `dictionaries' date from the 1960s already.
Most other studies support the idea that lenition is a highly salient\is{salience} feature.

For instance, lenition of /k/ had not (yet) spread to neighbouring West \isi{Wirral} in 1980: \textcite[97]{newbrook1999} recorded (heavily) fricated variants of this phoneme in only 8\% of cases.
In contrast to velar nasal plus, Liverpool lenition had thus not been taken over by speakers in West \isi{Wirral}.
The most probable explanation for the rejection of lenited variants is the stigma --- which presupposes \isi{salience} --- attached to them (while [ŋg] variants are largely below the radar).
Further evidence for the \isi{salience} of lenited /k/ variants can be found in \citealt{watsonclark2015}.
The authors ran a perception experiment where subjects had to rate speech samples representing different regional accents.
Since they were measuring perceivers' reactions in real-time it was possible to tease apart the impact that individual features had on the overall rating.
Occurrence of /k/ lenition caused a significant drop in the status rating of the speaker, which not only corroborates that this variable is salient\is{salience} (i.e. it was noticed), but also that it carries social meaning (low status).

	\section{Vowels}\label{sec.var.vow}

As explained in Section \ref{sec.var.con}, the Liverpool consonant system is phonologically identical to that of other Northern varieties or even English English in general.
Similarly, Scouse vowels have much in common with other Northern varieties in England.

Liverpool is north of the most important and probably also best known isogloss in England and so has the same vowel in words of the \textsc{strut} and \textsc{foot} lexical sets.
The most typical (and at least in working-class speech by far the most frequent) realisation is [ʊ].
Many middle-class speakers, however, tend to keep the two sets distinct.
This does not necessarily mean that middle-class speakers have [ʌ] in \textsc{strut} words.
Many speakers content themselves with ``merely making the vowel slightly different'' \citep[284]{knowles1973} and actual realisations usually range from a very slightly centralised [ʊ] to [ə].
Some confusion as to which vowel should be used in which words exists, and hypercorrect\is{hypercorrection}ions and mistakes occur (\citealt[286--287]{knowles1973} and \citealt[83]{knowles1978}).

Another issue where Scouse is in agreement with Northern English in general concerns [a] and [ɑː].
Liverpool English has [a] instead of RP [ɑː] in words like \emph{last, grass, bath} etc.
Middle-class speech again strives more towards RP but usually does not quite reach the target.
Typically, the resulting vowel is a compromise between [a] and [ɑː], both in terms of quality and duration and again there is some uncertainty and inconsistency (\citealp[cf.][287--289]{knowles1973} and \citeyear[83--84]{knowles1978}).
\citeauthor{watson2007} found that [ɑː] is generally used in \textsc{start} and \textsc{palm} words.
However, only women seem to really use the RP variant while men prefer a more fronted [a:] \parencite[cf.][358]{watson2007} --- much like the compromise described by \cite{knowles1973}.

Words like \emph{book} and \emph{look} have long [u:] instead of short [ʊ] in Liverpool.
\citet{knowles1973} described this pronunciation as being ``heard in the North Midlands from \isi{Merseyside} to beyond Leeds'', particularly in working-class speech \citep[290]{knowles1973}.
Often, long [u:] is centralised or fronted.
However, \citet[358]{watson2007} suggests this feature is fading, a statement the author of this study can (impressionistically) corroborate.
Only older speakers (roughly 60 years of age or older) seem to still have this vowel in \emph{book}.
It is also mostly this age group that makes use of [u:] as a typical accent feature in the imitation\is{accent performance} task (see Section \ref{sec.prod_method.interview}).

An aspect where Scouse is different from Northern English concerns the vowels in the lexical sets \textsc{face}, \textsc{price}, \textsc{goat}, \textsc{choice}, and \textsc{mouth}.
Unlike much of Northern England, Liverpool English has diphthongs in all these words, although \textsc{price} is occasionally monophthongised for some speakers \citep[cf.][358]{watson2007}.

		\subsection{happ\textsc{y}}\label{sec.var.vow.happy}

This section is concerned with happ\textsc{y}, i.e. the final vowel in words such as \emph{city}, \emph{baby}, \emph{pretty}.
With respect to RP, \citet[441]{harrington2006} writes that in the 1950s, the vowel used in this position was ``phonetically closer to [ɪ] in \textsc{kit} than to [i:] in \textsc{fleece}'', i.e. \emph{happy} was pronounced [hæpɪ], not [hæpi].
In the late 20\textsuperscript{th} century, however, happ\textsc{y} has undergone tensing in RP.
The phonetic realisation is now [i] for most speakers, and dictionaries generally use /i/ to represent this vowel.
Note that the \isi{change} was purely phonetic, not phonological, as [ɪ] and [i] do not distinguish meaning in the final unstressed syllables concerned.
Just like other \isi{change}s in RP during this period, happ\textsc{y}-tensing is associated with Estuary English \citep[cf.][]{wells1997}.
Lengthening of happ\textsc{y} to [iː] is probably due to the fact that, (a) in English, short vowels are not permitted word-final\is{phonological context}ly, and (b) happ\textsc{y} ``often occurs as the last syllable in a prosodic phrase, which is of course a primary context for synchronic lengthening (\ldots)'' \citep[441]{harrington2006}.

Although the standard pronunciation in modern RP is now clearly [i], some (very) conservative speakers might still adhere to the now outdated traditional (`upper-crust' in the terminology of \citealt{wells1982}) norm [ɪ] (as, e.g. \cite{trudgill1999} claims).
However, using Christmas broadcasts over a period of about 50 years \citet[cf.][452]{harrington2006} found that even Queen Elizabeth II had participated in the shift to a certain degree and moved her happ\textsc{y} vowel in the direction of the modern realisation.

Happ\textsc{y}-tensing has now spread to most parts of England, with the exception of ``[t]he Central North, Central \isi{Lancashire}, Northwest Midlands and Central Midlands areas''.
Here, the older pronunciation [ɪ] is still retained.
There are a few exceptions, though, namely the port cities Liverpool, Hull and Newcastle \citep[cf.][62]{trudgill1999}.
Liverpool, or rather the whole of \isi{Merseyside} (and parts of Chester) is therefore ``an `ee'-pronouncing island surrounded by a sea of accents which do not (yet) have this feature'' \citep[72]{trudgill1999}.
In fact, it is not clear whether other areas really will follow.
As a case in point, \textcite{flynn2010} has investigated happ\textsc{y} realisations of adolescents in Nottingham (which is part of Trudgill's `sea of accents' without happ\textsc{y}-tensing).
Not only did he find that lax happ\textsc{y} variants were holding their ground (although it has to be said that tense [i] variants are just as common), but also that particularly working-class females even used `hyper-lax' [ɛ] variants in a sizeable proportion of cases, presumably because they wish to actively distance themselves from tenser happ\textsc{y} realisations which are seen as `posh'.
Ultra-lax happ\textsc{y} realisations have also been attested for Sheffield \parencite{stoddartetal1999} and the Manchester area \parencite{watts2006}.

Given the above-mentioned `island status' of Liverpool, happ\textsc{y}-tensing is a distinguishing feature in the (supra-)regional context.
Like velar nasal plus, it had already spread across the Mersey to West \isi{Wirral} by 1980.
\textcite[97 and 99]{newbrook1999} in fact found ``Liverpool/general southern [i]'' to dominate clearly, with rates of occurrence around 83\% in informal speech registers, a \isi{change} which was apparently driven by younger females, who were among the first to introduce Liverpool variants of this and several other variables.

Notwithstanding its usefulness as a feature that distinguishes Liverpool English from surrounding non-standard accents, happ\textsc{y}-tensing seems to have low \isi{salience} and is not the subject of comments about Scouse (in Liverpool itself, and also in West \isi{Wirral}) --- possibly because it does not diverge from the modern standard.

		\subsection{\textsc{nurse} --- \textsc{square}}\label{sec.var.vow.nurse}

As another ``\isi{Merseyside} feature'' \citet[72]{trudgill1999} notes the \textsc{nurse}-\textsc{square} merger\footnote{Patrick Honeybone (p.c.) is critical of calling this feature a merger because the term either implies \enquote{speakers are actively/synchronically abandoning a contrast, or at least that this is a merger which has happened in the history of Scouse [as opposed to before the formation as a new dialect]}, neither of which he considers to be true. I tend to agree, but, for reasons of convenience, have decided to follow other studies \parencite{trudgill1999,watsonclark2013} in using the label \enquote*{merger} nonetheless.}, i.e. the fact that words such as \emph{fair} and \emph{fur}, or \emph{purr} and \emph{pair} can be (near-) homophones in Liverpool English.
In older, very traditional Liverpool English, this merger used to be centralised \parencite[cf.][323]{west2015}, much like in the surrounding areas, but this is no longer the case.
\citet[cf.][68 and 71]{delyon1981} distinguishes 15 possible realisations for \textsc{nurse} and 18 for \textsc{square} in her auditory analysis, but the most typical realisation (in a broad Scouse accent) for both vowels is [ɛː] or [eː], sometimes even reaching [ɪː] \citep[cf.][358]{watson2007}.
\citet[127]{honeybone2007} mentions the same range of realisations (``central and front vowels''), but calls the front vowels in particular ``very robust'' and gives [skwɛː] \emph{square} : [nɛːs] \emph{nurse} as examples.

According to \citet[358]{watson2007}, \textcite{delyon1981} does not succeed in giving a (quantitative) description of how these variants are socially distributed (as they can be expected to be).
Given his own reservations about the scope of his study (cf. Section \ref{sec.var.general}), \citet{knowles1973} does not fare much better, but his thesis does contain a number of remarks about the subject.
For example --- although this is not very exciting news --- he states that, generally, the working-class residents of Vauxhall do not make this distinction, whereas the middle-class speakers from Aigburth usually do, with the Aigburth women topping the list (which is, this time, in line with most research on gender differences that followed).
The degree of difference between the two vowels can, however, be very subtle, to the point that ``a gesture towards the \isi{prestige} standard'' (for the speaker), ``may be for the hearer just another variant of a dialect vowel'' \citep[cf.][295--297]{knowles1973}.

At the same time he claims that the ``typical middle-class vowel is /ɜ̟/ or the RP type /ɜ/''. He reports working-class speakers as using mostly [ɛ̈] (``further forward on the axis'') and explains that younger speakers have an even more fronted (and raised) [ë] \citep[271]{knowles1973}.
He adds that /ɜ/ in particular ``merits further study for various age-groups and in various parts of \isi{Merseyside}'' \citeyearpar[320]{knowles1973}.

Concerning possible sources of this merger, \citet[128]{honeybone2007} mentions the dialect of South \isi{Lancashire} as the most obvious candidate.
He attests ``a similar lack of contrast'' there but stresses the fact that although the same two vowels as in Liverpool are concerned, the direction of the merger is different.
Where Scouse merges \textsc{nurse} and \textsc{square} towards front vowels [ɛː] or [eː], South \isi{Lancashire} English has a central vowel, ``such as [ə\textsuperscript{ɹ}: \(\sim\) ɜː] (with residual rhoticity still an option)''.

\textcite{honeybone2007} also lists a number of studies reporting similar mergers in several \isi{Irish} varieties.
\citet{wells1982}, for instance, tells us that Belfast English has a merger very similar to that of Scouse, realising \emph{fair, fir, fur} all as [fɛːɹ].
\citet[cf.][48]{harris1985} describes a merger comparable to the one in South \isi{Lancashire} for urban speakers of Lagan Valley (in Northern Ireland).
The vowel used is a central [ɜː] (his examples are [dɜːɻ] \emph{dare}, and [stɜːɻ] \emph{stair}).
Talking about `fashionable Dublin English' \citet{hickey1999} asserts that \textsc{nurse} and \textsc{square} have the possible realisations [nə\textsuperscript{ɻ}s] and [skwə\textsuperscript{ɻ}] respectively.
\textcite[128]{honeybone2007} points out that these are statements about current (or comparatively recent) stages of the respective dialects and that it is somewhat speculative to assume that ``these patterns can be extrapolated to the varieties of Hiberno-English which were spoken in Liverpool at the time of koineisation''.
In fact, given the intense and long-lasting contacts between Liverpool and Ireland\is{Irish} it is just as possible that the merger actually crossed the \isi{Irish} Sea westwards instead of eastwards.

What these reports do show, however, is that ``the pre-\emph{r} vowels in these words are susceptible to considerable variation in Hiberno-English varieties (\ldots) in ways which would have differed from those supplied by \isi{Welsh}, \isi{Scottish} and most English dialects during koineisation'', which is why \isi{Irish} influence does seem plausible \citep[128]{honeybone2007}.
\textcite{honeybone2007} still stresses the fact that the most important donor variety with regard to the \textsc{nurse}-\textsc{square} merger must have been \isi{Lancashire} English --- ``where there was a complete lack of contrast'' --- and that \isi{Irish} varieties only provided a further push towards adapting this feature which was already in the pool \citep[129]{honeybone2007}.
Just as with the Liverpool lenition pattern it has to be noted that the \textsc{nurse}-\textsc{square} merger was not borrowed wholesale from \isi{Lancashire} English or any other variety and simply carried on. 
Rather, it was actively selected from ``the mix\is{new-dialect formation} of \isi{dialect contact}'', adapted, \isi{change}d, and made a part of Liverpool English \parencite[cf.][129]{honeybone2007}.
As has been observed in the case of /k/ lenition, merged \textsc{nurse}-\textsc{square} realisations were not commonly found on the other side of the Mersey in 1980.
\textcite[95]{newbrook1999} reports that 2 out of 3 speakers in his sample maintain a difference between these two vowels.
At the same time both vowels seem to exhibit ``surprisingly low \isi{salience}'' and definitely less ``than elsewhere in \isi{Merseyside}'' (where \isi{salience} must thus be higher) \parencite[95]{newbrook1999}.

\citet[cf.][128]{honeybone2007} asserts that the \textsc{nurse}-\textsc{square} merger as it is found in Liverpool is not known to exist in any other variety in England, Scotland\is{Scottish}, or Wales\is{Welsh}.
This is not quite correct, though, as in Teesside ``the same merger between the vowels of \emph{hair} and \emph{her} which is found in Liverpool (\dots)'' is attested \parencite[70]{trudgill1999}.
While not unique to Liverpool, this merger is in any case rare enough to be generally perceived as one of the most characteristic (or even defining) and most salient\is{salience} features of Scouse, to the point where it is commonly picked up by comedians and the like --- an early example is Ken Dodd's catchphrase `Whaire's me shairt?' \citep[cf.][73]{trudgill1999}.
The \textsc{nurse}-\textsc{square} merger also figures prominently in the Scouse phrase books pioneered by Frank Shaw and Fritz Spiegl.
In fact, these two vowels are the ones that most often occur with a non-standard spelling in the \emph{Lern Yerself Scouse} series \parencite[cf.][322]{honeybonewatson2013}.
The particular re-spellings that are chosen for words which are minimal pairs in RP hint at ``an aware\is{awareness}ness of the fact that these words in these lexical sets can be pronounced in the same way'' \parencite[324]{honeybonewatson2013}, while the high \isi{frequency} with which this is done indicates that \textsc{nurse} and \textsc{square} ``are imbued with local meaning'' and constitute the ``most salient\is{salience}'' of the vocalic features the authors analysed.

Additionally, this feature is the second for which perceptual data are already available (cf. Section \ref{sec.var.con.len}).
\textcite{watsonclark2013} played recordings to subjects and asked them to rate the audio clips (again in real-time) with respect to how `posh' the speaker sounded.
He naturally produced merged \textsc{nurse} and \textsc{square} vowels with central realisations.
In addition, fronted (Liverpool-like) variants were re-synthesis\is{resynthesis}ed and participants were randomly assigned to one of two guises, which corresponded to 100\% central and 100\% fronted realisations, respectively \parencite[cf.][305--306]{watsonclark2013}.
Listeners from St. Helens and Liverpool reacted to non-standard realisations of both vowels (front \textsc{nurse} and central \textsc{square}) by assigning lower status values to the speaker --- at least when non-standard variants preceded standard ones in the audio clip.
This corroborates that the \textsc{nurse}-\textsc{square} merger is ``a salient\is{salience} feature of English in north-west England'' \parencite[cf.][317--320]{watsonclark2013}.

	\section{Summary}

I have tried to give a (very) short overview of the features that constitute the Liverpool accent in this chapter.
Special mention has been made of the four variables whose production and perception will be the focus of the rest of this book.
They are: velar nasal plus and lenition of /k/ for the consonants, and happ\textsc{y}-tensing and the \textsc{nurse}-\textsc{square} merger for the vowels.
In the literature, velar nasal plus and happ\textsc{y}-tensing are (implicitly) counted among the less salient\is{salience} features, whereas lenition and the \textsc{nurse}-\textsc{square} merger are said to form part of the \isi{stereotype} of Scouse.
This received, and comparatively broad, distinction into salient\is{salience} and non-salient\is{salience} variables constitutes the starting point and the basis of the present study, and will be updated and refined in the following chapters.