\chapter{Reading passage}
\label{app.reading}
When you get older, your childhood often seems like the best time you ever had. We all remember little things that aren't really important but which still appear to have left their mark.

Me, my best mate John and a bunch of other kids used to meet in the little square next to the brick church. Except for the water fountain, there wasn't anything special about it, but for us it was the best place on earth, because no one cared what we did there.

Some of the other kids were already real characters at the time. Bill, for instance. We loved to play hide-and-seek, but Bill would always turn around early and peek. He used to say he had miscounted. It wasn't fair and we all knew it was a flimsy excuse, but he got very angry when we told him so. He could be a real trouble maker, so we just took to hiding a bit quicker when it was his turn. He was fun all the same: we once dared him to go home naked --- and lost.

There were also a number of girls, but at the time we didn't consider them part of the gang. Susan was the bookish one, rather quiet though physically fitter than most of us, while Jenny, as a natural born public speaker, was giving speeches all the time.

Sometimes we just chatted. One day, we spent hours going on about what we wanted to become when we were grown-ups. Most of the girls either wanted to be a vet or dreamed of being a famous singer. John wanted nothing but to be a baker (he really did love cake), and Bill had his mind set on becoming a snooker champion.
Later that day, Susan arrived carrying a cardboard box.

The box was full of kittens, a whole litter. ``Oh look, baby cats!'', Jenny squeaked, ``take one, John!'' ``No, thanks. I don't like animals'', John answered. ``Well, I think you're just scared of that little fang'', said Jenny. That was a bit of a clanger, because John had been hurt quite badly by a dog as a toddler and he was still rather touchy about the subject.

To take attention off the matter, I took one of the cats and held it close to my face. It started to purr immediately and licked my nose with its rough little tongue.
When I got home, I kicked off my sneakers and found my mum brushing her hair in the bathroom.

She looked disapprovingly at my scruffy clothes (I had ruined my shirt again), but the cat quickly caught her attention. My parents weren't stingy but I knew room was scarce in our house. So I said: ``Look, mummy, I found a kitty! Please, can I keep it as a pet? It could live in the cupboard under the stairs.''

I felt rather canny at the time for coming up with that idea, although, in retrospect, I probably stole it from a children's book. She seemed reluctant at first, but then she stroked the cat's fluffy fur and, after a longish silence, she sighed: ``Ok\dots, but you'll have to care for it yourself!''

Some stuff might indeed have been better in the days when I was young and strongish. Back then, that furry little thing was all it took to make me happy.