\chapter{Introduction}
\label{ch.intro}

\section{Intentions --- what this study is about}
\label{sec.intro.intent}

The present dissertation is primarily interested in the impact that sociolinguistic \isi{salience} can have on the perception of language.
As such, it is firmly rooted within sociophonetics, but also inherently inter-disciplinary in nature due to the fact that mental representations, cognitive processing, and the influence of \isi{stereotype}s are relevant in the context of the research question.
A number of studies conducted in recent years have shown that perceivers integrate social information about speakers when processing linguistic material.
\textcite{niedzielski1999} and \textcite{hayetal2006a} in particular provide evidence that subjects perceive one and the same acoustic stimulus differently depending on what they sub-conscious\is{awareness}ly believe to know about the speaker they are listening to.
\textcite{haydrager2010} then went one step further and showed that even cues that are both more subtle and more indirect are capable of biasing the cognitive system towards processing or, more precisely, categorising linguistic input in a particular way.
These data are not only extremely relevant for models of how humans cognitively deal with variation in language, but especially the results of \textcite{haydrager2010} additionally have the potential of changing the way linguistic experiments are designed and conducted: if even small objects completely unrelated to the task can influence the outcome of an experiment by their mere presence, then it seems necessary to control for the physical surroundings of such experiments much more carefully than most of us probably have done so far.

There is, however, an aspect that has not figured prominently in previous research and that might be able to qualify the conclusions drawn from these studies: \isi{salience}.
In recent years, most sociophoneticians have incorporated some form of episodic memory\is{memory structure} in their theoretical frameworks, and this is also the model that is best able to explain the results derived from previous \isi{priming} studies in sociolinguistics.
Within this framework, \isi{salience} should actually play a crucial role for \isi{priming} effects because salient\is{salience} sensory events are believed to dominate long-term memory\is{memory structure} due to their prominence in perception \parencite[cf.][]{pierrehumbert2006}.
It is only logical that they should then also be more prone to manipulations such as \isi{priming}, which leads to the main hypothesis of this study: the strength of an \isi{exemplar} \isi{priming} effect is a direct function of the sociolinguistic \isi{salience} of the test variable.
Priming\is{priming} effects of the kind that \textcite{niedzielski1999} and \textcite{hayetal2006a} found would then be restricted to linguistic variables that are highly salient\is{salience}, possibly even to those that have reached the level of conscious\is{awareness} aware\is{awareness}ness in the relevant \isi{speech community} (\emph{\isi{stereotype}s} in Labovian terminology).

The testing ground for this hypothesis is Scouse, the variety of English spoken in the city of Liverpool and parts of its immediate surroundings in the north-west of England.
There are several points which make Liverpool English a good candidate for the present study:
\begin{inparaenum}[(1)]
	\item It has a number of phonological features (some more, some less salient\is{salience} according to the literature) that set it apart from the standard and surrounding non-standard varieties;
	\item It is one of the most widely known \parencite[cf.][]{trudgill1999}, and
	\item most heavily stigmatise\is{stigmatisation}d varieties in the UK \parencite[cf.][]{montgomery2007}.
\end{inparaenum}
Scouse is a convenient choice of variety in the context of this thesis because the presence of variants that attract overt commentary\is{overt commentary} is obviously a prerequisite for testing the hypothesis formulated above.

Four phonological variables (two vocalic, two consonantal) have been selected as the focus of this thesis: happ\textsc{y}-tensing, velar nasal plus, the \textsc{nurse}-\textsc{square} merger, and lenition of /k/.
The first two of these are generally thought to carry very low levels of social \isi{salience} in Liverpool, while the remaining two are considered to be \isi{stereotype}s by many linguists.
However, there are a number of reasons that advise against blindly and exclusively categorising these variables as salient\is{salience} or non-salient\is{salience} on the basis of previous research alone.
The most important of these is that, for the present study, it is desirable to have a classification that is more fine-grained than the binary salient\is{salience} vs. non-salient\is{salience} one.
Additionally, Liverpool English is reported to go against the general trend of \isi{dialect levelling} found in many other places \parencite{kerswill2003}.
Instead, \textcite[237]{watson2007a} found Scouse to be ``getting Scouser'', at least with respect to some variables.
Especially against the backdrop of this ongoing \isi{change}, it is therefore necessary to independently ascertain the \isi{salience} of the four variables under scrutiny here first.
This is done by analysing production data (collected in the form of sociolinguistic interviews) and measuring the \isi{salience} of a variable with respect to the traditional \isi{indicator}-\isi{marker}-\isi{stereotype} hierarchy introduced by Labov.

This approach provides the opportunity to address several additional questions along the way, as it were, such as whether younger Liverpudlians have stronger local accents than older speakers in \emph{every} respect, or how these \isi{change}s are related to local \isi{identity}, the internal as well as external \isi{image} of their city, and \isi{attitude}s of speakers towards their variety.
These issues are, of course, particularly interesting in the case of Liverpool, because the city has seen such a tremendous amount of physical, economic, and social \isi{change} in the last 50 years, and this is likely to have at least some impact on the (socio-)linguistic behaviour of speakers.
Furthermore, Liverpool English is a variety for which \textcite[351]{watson2007} stated in 2007 that ``modern research [was] lacking'', especially in the area of variation along social dimensions such as age, gender, or class.
It is true that, in the 11 years since Watson's claim, a number of linguistic studies focusing on Liverpool have been published, but I would still argue that we know far more about many other varieties of English than we do about Scouse.
As far as I am aware, for instance, there is still no complete descriptive account of Liverpool English except \cite{knowles1973}, which is now quite dated and also clearly and explicitly \emph{not} a truly variationist study of the kind \textcite{watson2007} refers to.
I will try to narrow this gap a bit, but it should be noted that the primary purpose of analysing production data, in the present study, is to provide a sound basis for comparison for the subsequent perception test.
The focus is therefore on establishing the \isi{salience} of the four test variables and on discovering any differences (with respect to \isi{salience}) between social groups, particularly along the age dimension.

\section{Restrictions --- what this study is not about}
\label{sec.intro.restrict}

An a priori limitation of my thesis is that it is only concerned with Scouse as an accent.
Local characteristics in the lexicon, (morpho-)syntax, or discourse pragmatics will remain unaddressed.
It is also \emph{not} the aim of this book to be an updated version of \citeauthor{knowles1973}'s \citeyear{knowles1973} study and provide a complete description of the phonological system of Scouse.
Rather, it focusses (almost) exclusively on the four variables listed above and largely ignores other segmental and suprasegmental features of Liverpool English.
A detailed account of the \isi{social stratification} of local variants is equally beyond the scope of my thesis.
Social differentiations of subjects (for the production data) are therefore comparatively coarse, and the size of the speaker sample does not permit much more fine-grained distinctions.
It is, however, more than sufficient for assessing the \emph{social \isi{salience}} of our variables, which is the purpose it was collected for.

This brings me to the second issue that it might be preferable to clarify from the very beginning of this book.
Despite the fact that \emph{\isi{salience}} appears in the title of this work and notwithstanding that the term will turn up again and again in what is to follow, the present study is \emph{not} a book \emph{about} \isi{salience} per se (cf. Chapter \ref{ch.sal}).
There is an ongoing debate among researchers about what exactly \isi{salience} is or what precisely it should refer to.
My analysis will not add anything to this discussion, mostly because I am not interested --- in the context of the present thesis --- in what \emph{makes} something salient\is{salience}.
Instead, I intend to address the question of what \isi{salience} \emph{does} in perception, particularly when \isi{priming} is involved.
In other words, the spotlight is on the \emph{effects} of \isi{salience}, not on its \emph{causes}.
Essentially, social \isi{salience} will be the scale used to measure the degree of aware\is{awareness}ness of, and \isi{attention} paid to, a particular variable.
I will then show that the level of aware\is{awareness}ness correlates with the strength of the \isi{priming} effect.
How and why aware\is{awareness}ness came about in the first place is irrelevant for this purpose and will not be discussed any further.

\section{Structure of the book}
\label{sec.intro.structure}

Chapter \ref{ch.hist} sketches the history of the city of Liverpool and its accent to give the reader an idea about the social \isi{change}s that have taken place in this city and how they might influence the \isi{attitude}s of speakers from different generations towards Scouse and questions of local \isi{identity}.
Chapter \ref{ch.var} contains a short overview of the pool of phonetic and phonological features that Liverpool English draws from, and presents the four variables that this book focusses on.
Chapter \ref{ch.sal}, finally, explains how the term \emph{\isi{salience}} is used in this work, and also how it will be operationalised.
Furthermore, it lays out some fundamental principles of \isi{exemplar} theory and describes how the main hypothesis of this dissertation is motivated by the theoretical framework.

Next is a a comprehensive description (Chapter \ref{ch.prod_method}) of how the production data were collected (interview structure, sampling), measured (parameters, semi-automatic processing), and analysed (\isi{normalisation}, statistical modelling).
Chapters \ref{ch.prod_results_vow} (vowels) and \ref{prod.res.con} (consonants) contain the quantitative analysis of the data gathered from the sociolinguistic interviews, while Chapter \ref{prod.res.qual} presents a recapitulatory qualitative analysis of participants' explicit comments\is{overt commentary} about (specific features of) their accent, local \isi{identity}, and the like.
In Chapter \ref{ch.prod_discussion}, both quantitative and qualitative results are summarised, discussed, and contextualised.
While this part dominates in terms of the space devoted to it, this should not be taken to imply that it is also conceptually more important --- it just so happens that a detailed analysis of production patterns is rather space and time consuming, even when it is a comparatively restricted one.

In the remaining chapters, this book turns to perception.
Stimulus generation, recruitment of participants, presentation of test material and other methodological issues are treated in Chapter \ref{ch.perc_method}, while the results of the online perception test are reported in detail in Chapter \ref{ch.perc_res}.
My interpretation of said results (Chapter \ref{ch.perc_disc}) takes into account both the production data, on the one hand, and previous research, particularly by \textcite{hayetal2006a} and \textcite{haydrager2010}, on the other.
Chapter \ref{ch.conclusion}, finally, rounds off the study with a brief recapitulation of the most relevant findings and conclusions.

Most chapters end with a summary that contains the main points.
Exceptions to this rule are the chapters on methodology and the ones presenting the results of the quantitative and qualitative analyses.
In the former case, a summary was deemed to be rather unnecessary as the whole point of these chapters is to describe the methods employed \emph{in detail} for reasons of replicability.
The `results' chapters, on the other hand, are summarised in the discussions (Chapters \ref{ch.prod_discussion} and \ref{ch.perc_disc}), and therefore do not require a résumé of their own.