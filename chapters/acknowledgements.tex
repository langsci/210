\chapter*{Acknowledgements}

This work owes a lot to the many colleagues and friends who directly or indirectly contributed to its making and I offer my sincere apologies to anyone I might have forgotten in the list below.
First and foremost, I would like to thank my supervisor Bernd Kortmann, whose door was always open, and without whom I would never have got interested in Liverpool English in the first place.
I am equally indebted to Peter Auer, Brigitte Halford, Patrick Honeybone, and two anonymous reviewers, whose comments were very helpful in improving this book.
Thanks are also due to a plethora of other people whose help was greatly appreciated:
Christian Langstrof was a tremendous source of knowledge for anything related to phonetics, Praat, and vowel normalisation.
Alice Blumenthal-Dramé and Verena Haser answered numerous questions about statistics and R.
As my first, invaluable, contact in Liverpool, Michael Pace-Sigge told me where best to find Scousers to interview.

Among many others, David Brazendale, Amanda Cardoso, Marije Van Hattum, Michaela Hejna, Victorina Gonzalez-Diaz, Sofia Lampropoulou, Ian McEvoy, Linda McLoughlin, Katja Roller, Erik Schleef, and Kevin Watson (in alphabetical order) all helped me to recruit participants in one way or another.
My heartfelt thanks go out to everyone who took the time to give me an interview, or who spared half an hour to take part in the online test.
Without the data they provided this thesis would not have been possible.

I am equally indebted to Danielle Turton, who --- although a proud Mancunian --- consented to record Scouse lenition stimuli for the perception experiment.
Andrew MacFarlane also deserves a mention here, as it was his ``Herr Hitler'' mnemonic that set me on the most promising track to explaining parts of my perception data.
My mother Gertraude and my brother Maiko have my gratitude for going through the horror of proofreading an early version of this manuscript.
All remaining errors are my own.

During the four years it took me to complete this thesis my wife Daniela was my anchor and my sail.
She shared my enthusiasm when things were going well, and provided comfort and encouragement when they were not.
Her influence and support are in every word.